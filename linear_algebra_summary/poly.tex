\section{一元多项式}
设$p(x)$为$n$次实系数多项式,则$p(x)$在$\ma{R}[x]$中的不可约多项式只能是一次和$\Delta<0$的二次式,因而$p(x)$的实根个数可能小于$n$, 这使得处理问题不方便,因而常常将$p(x)$放在$\ma{C}[x]$中来研究。比如,考虑多项式$x^d-1$什么时候能够实整除$x^n-1$的时候,代入复根就很方便。

因而一般地考虑$p(x)\in F[x]$,域$F$有扩域$E$,往往由于$F$过小使得其性质不够好,需要通过研究$p$在$E[x]$中的行为来反推$p$在$F[x]$中的行为。最典型的是下面的定理
\begin{thm}
	设$F$是一个域,$E$为它的扩域,对于$F[x]$中任意两个多项式$f(x),g(x)$有
	\begin{description}
		\item[(a)] $g|_F f$当且仅当$g|_E f$,
		\item[(b)] $(f,g)_F=(f,g)_E$,
		\item[(c)] $f$在$F[x]$中有重因式当且仅当$f$在$E[x]$中有重因式.
	\end{description}
	其中$|_F$表示在$F[x]$中整除,$(\,,)_F$表示在$F[x]$中的首一最大公因式.
\end{thm}
\begin{proof}
	处理整除问题最常用的办法有三个,带余除法的唯一性,不可约多项式与唯一不可约分解定理。
	\begin{description}
		\item[(a)] 必要性显然. 现在假设$g|_E f$,考虑$f,g$在$F[x]$中的带余除法
		\[f(x)=q(x)g(x)+r(x),\]
		其中$q,r\in F[x]$, $\deg r<\deg g$. 这个带余除法可以直接看成$E[x]$中的带余除法,由于$E[x]$中带余除法
		唯一,故而$r(x)=0$, 即$g|_F f$.
		\item[(b)] 在$F[x]$中将$f,g$做辗转相除法,其最后一个不为零的余式首一化后记为$d_F(x)$. 由于整个辗转相除法可以看成在$E[x]$中进行,而辗转相除法不过是一系列带余除法的组合,因而在$E[x]$中辗转相除法唯一,所以得到的最后一个不为零的首一化余式$d_E(x)=d_F(x)$. 由于
		\[d_F=(f,g)_F,\quad d_E=(f,g)_E,\]
		就得到证明.
		\item[(c)] 由于$f$在$F[x]\,(E[x])$中有重因式当且仅当$(f,f')_F\neq 1\,((f,f')_E\neq 1)$, 由(b)立知正确.
	\end{description}
	可以看出,证明的关键在于带余除法(辗转相除法)的唯一性和要先在小域里作.
\end{proof}
事实上,对$f$在大小域的重因式,有下面更好用的结论,且证明也可以更直接.
\begin{thm}	
	假设$f\in F[x], F\subset E,\deg f>0$在$F,E$中分别有不可约分解
	\begin{align*}
		f(x)&=p_1(x)^{r_1} p_2(x)^{r_2}\dots p_n(x)^{r_n},\quad p_i\in E[x],\\
		f(x)&=q_1(x)^{s_1} q_2(x)^{s_2}\dots q_m(x)^{s_m},\quad q_i\in F[x].
	\end{align*}
	其中$r_i,s_i\geq 1,\deg p_i>0,\deg q_i>0$, $p_i,q_i$分别在$E,F$上不可约且$(p_i,p_j)=(q_i,q_j)=1$, 对任意$i\neq j$. 那么$m\leq n$,
	\[\{r_1,r_2,\dots,r_n\}=\{s_1,s_2,\dots,s_m\}\]
	且$m=n$当且仅当$p_i\in F[x]$对所有的$1\leq i\leq n$.
\end{thm}