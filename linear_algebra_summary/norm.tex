\section{实复正规矩阵的通性与特性}
\subsection{实正规矩阵正交准对角化}
实对称阵,实斜对称阵,实正交阵都是广泛应用的矩阵,他们都可以正交准对角化,事实上所有的实规范矩阵都可以正交准对角化. 复矩阵也有类似的概念,定义$A^*=\overline{A}^T,\, A\in \ma{C}^{n\times n}$, 称$A$为Hermite阵(复对称)若$A=A^*$, 为
斜Hermite阵(复斜对称)若$A^*=-A$,为酉阵(复正交)若$AA^*=A^*A=I$, 为复正规阵若$AA^*=A^*A$. 为方便,若省略实复,都指实阵,又称复阵$A$的特征多项式的复根为其特征复根,下面先给出实阵的特性. 为了能将实正规阵做准对角分解,下面的引理是必要的.
\begin{lem}\label{lem:2sob}
	设$A$为实正规阵,把它看成复正规阵,设$\la\in\ma{C},\al\in\ma{C}^n$为$A$的任一特征值与相应的特征向量,又拆$\al=\beta+i\gamma,\;\beta,\gamma\in \ma{R}^n$, 则
	\[A^T\al=\overline{\la}\al,\quad (\beta,\gamma)=0,\quad \|\beta\|=\|\gamma\|,\]
	其中$(x,y)=x^Ty,\,\|x\|=\sqrt{(x,x)},\,\forall x,y\in\ma{R}^n$.
\end{lem}
\begin{thm}
	设$\la_{\pm 1},\dots,\la_{\pm r},\mu_1,\dots,\mu_{n-2r}$为$n$级实规范阵$A$的全部特征复根,其中
	\[\mu_j\in\ma{R},\;\la_{\pm j}=a_j\pm i b_j,\;a_j\in\ma{R},\;0\neq b_j\in \ma{R},\]
	则存在实正交矩阵$U$,
	\begin{equation}\label{eq:norm}
	U^T A U=\diag\left(\mu_1,\dots,\mu_{n-2r},
	\block{a_1}{b_1},\dots,\block{a_r}{b_r}\right).
	\end{equation}
\end{thm}
\begin{proof}
	用归纳法证明最方便,$n=1$已经成立,现在归纳$n$. 
	\begin{description}
		\item[(a)] 如果$A$有特征值(即有特征复根恰为实数)$\mu_1\in\ma{R}$, 则取单位向量$\eta_1$使得$A\eta_1=\mu_1 \eta_1$. 将$\eta_1$扩展为$\ma{R}^{n\times n}$的一组标准正交基
		\[P=(\eta_1,\eta_2,\dots,\eta_n)\]
		则$PP^T=I$且
		\[AP=(\mu_1\eta_1,A\eta_2,\dots,A\eta_n)=(\eta_1,\eta_2,\dots,\eta_n)
		\begin{pmatrix} \mu_1 & \al\\ 0 & B\end{pmatrix}.\]
		化为
		\begin{equation}\label{eq:norminduction}
		P^T AP=\begin{pmatrix} \mu_1 & \al\\ 0 & B\end{pmatrix}.
		\end{equation}
		由于$A$实正规,故$P^T AP$实正规,猜想\eqref{eq:norminduction}的右端可以解出$\al,B$的性质.
		事实上由于
		\begin{align*}
		\begin{pmatrix} \mu_1 & \al\\ 0 & B\end{pmatrix}
		\begin{pmatrix} \mu_1 & 0\\ \al^T & B^T\end{pmatrix}&=
			\begin{pmatrix} \mu_1^2+|\al|^2 & \al B^T\\ B\al^T & BB^T\end{pmatrix},\\
		\begin{pmatrix} \mu_1 & 0\\ \al^T & B^T\end{pmatrix}
		\begin{pmatrix} \mu_1 & \al\\ 0 & B\end{pmatrix}&=
		\begin{pmatrix} \mu_1^2 & \mu_1\al\\ \mu_1\al^T & \al^T\al+B^T B\end{pmatrix}.
		\end{align*}
		比较可知$\al=0,\, BB^T=B^TB$. 显然$B$的所有特征复根为
		\[\la_{\pm 1},\dots,\la_{\pm r},\mu_2,\dots,\mu_{n-2r},\]
		故按假设$B$可以靠正交阵$P_1$准对角化为
		\[P_1^T B P_1=\diag\left(\mu_2,\dots,\mu_{n-2r},
		\block{a_1}{b_1},\dots,\block{a_r}{b_r}\right).\]
		只要令$U=P\diag(1,P_1)$就得到\eqref{eq:norm}.
		\item[(b)]
		如果$A$没有特征值,取$0\neq \beta\in \ma{C}^n$满足$A\beta=\la_1 \beta$. 拆$\beta=\beta_1+i\beta_2$, 其中$\beta_j\in\ma{R}^n$. 则由引理\ref{lem:2sob}知$\beta_{1,2}$相互正交,等长且
		\[A(\beta_1,\beta_2)=(\beta_1,\beta_2)\block{a_1}{b_1}.\]
		这已经有了雏形,不同在于$\beta_{1,2}$不是单位长,这可以通过同时除以$\|\beta_{1,2}\|$来实现. 即记
		$\eta_{1,2}=\beta_{1,2}/\|\beta_{1,2}\|$,将其扩展成一个标准正交基$P=(\eta_1,\eta_2,\dots,\eta_n)$
		,则$PP^T=I$且
		\[
		AP=(\eta_1,\eta_2,\dots,\eta_n)
		\begin{pmatrix} C & \al\\ 0 & B\end{pmatrix},\quad C=\block{a_1}{b_1}.
		\]
		化为
		\begin{equation}\label{eq:2sobform}
		P^TAP=\begin{pmatrix} C & \al\\ 0 & B\end{pmatrix}.
		\end{equation}
		同样需要利用$P^TAP$的正规性,
		\begin{equation}\label{eq:break}
		\begin{aligned}
			\begin{pmatrix} C & \al\\ 0 & B\end{pmatrix}
			\begin{pmatrix} C^T & 0\\ \al^T & B^T\end{pmatrix}&=
			\begin{pmatrix} CC^T+\al\al^T & \al B^T\\ B\al^T & BB^T\end{pmatrix},\\
			\begin{pmatrix} C^T & 0\\ \al^T & B^T\end{pmatrix}
			\begin{pmatrix} C & \al\\ 0 & B\end{pmatrix}&=
			\begin{pmatrix} C^TC & C^T\al\\ \al^TC & \al^T\al+B^TB\end{pmatrix}
		\end{aligned}
		\end{equation}
		比较可知$\tr(\al\al^T)=0$, 从而$\al=0$并且$BB^T=B^TB$. 显然$B$的所有特征复根为
		\[\la_{\pm 2},\dots,\la_{\pm r},\mu_1,\dots,\mu_{n-2r},\]
		故按假设$B$可以靠正交阵$P_1$准对角化为
		\[P_1^T B P_1=\diag\left(\mu_1,\dots,\mu_{n-2r},
		\block{a_2}{b_2},\dots,\block{a_r}{b_r}\right).\]
		只要令$U=P\diag(I_2,P_1)Q$其中$Q$为合适的两行(列)对换初等阵,就得到\eqref{eq:norm}.
		\item[(c)] 事实上若不直接用引理\ref{lem:2sob},也可以由$\beta_{1,2}$线性无关去作Schmidt正交归一化为$\eta_{1,2}$. 再记$A$作用其上的矩阵为$C$,写出\eqref{eq:2sobform}和\eqref{eq:break},再解出$\al=0$. 至于$C$的具体形式,可以把它的四个元素都设出来,再利用$CC^T=C^TC$和$C$的特征复根为$\la_1$去暴力解,就可解出.
	\end{description}
\end{proof}
终于,对于具体的实正规阵,有了下面的定理
\begin{thm}
	设$A$为实规范矩阵,则
	\begin{description}
	\item[(1)] $A=A^T$当且仅当$A$的全部特征复根为实数,当且仅当$A$能够实正交对角化.
	\item[(2)] $A=-A^T$当且仅当$A$的全部复特征根的实部为$0$,当且仅当存在实正交阵$U$,
	\[U^T A U=\diag\left(0,\dots,0,
	\block{0}{b_1},\dots,\block{0}{b_r}\right),\]
	其中$b_j\neq 0$.
	\item[(3)] $AA^T=I$当且仅当$A$的全部特征复根的模为$1$,当且仅当存在实正交阵$U$,
	\[U^T A U=\diag\left(\pm 1,\dots,\pm 1,
	\block{\cos\cta_1}{\sin\cta_1},\dots,\block{\cos\cta_r}{\sin\cta_r}\right),\]
	其中$0<\cta_j<\pi$.
	\end{description}
\end{thm}
\subsection{复正规矩阵正交对角化}
用和实正规阵类似的办法,可以得到任意的复正规阵都可复正交对角化,之所以能有这么河蟹的结果当然还是因为我们的代数学基本定理. (其实任意复矩阵都可以复正交上三角化也归功于它)
\begin{thm}
	设$A$为复正规阵,则存在酉阵$U$使得
	\[U^* AU=\diag(\la_1,\la_2,\dots,\la_n)\]
	其中$\la_j$为$A$全部特征值. 进一步的
	\begin{description}
	\item[(1)] $A=A^*$当且仅当$A$的特征值都是实数.
	\item[(2)] $A=-A^*$当且仅当$A$的特征值的实部都为$0$.
	\item[(3)] $AA^*=I$当且仅当$A$的特征值的模都为$1$.
	\end{description}
\end{thm}
\begin{proof}
	任取$A$的特征值$\la_1\in\ma{C}$, 则取单位向量$\eta_1$使得$A\eta_1=\la_1 \eta_1$. 将$\eta_1$扩展为$\ma{C}^{n\times n}$的一组标准正交基
	\[P=(\eta_1,\eta_2,\dots,\eta_n)\]
	则$PP^*=I$且
	\[AP=(\la_1\eta_1,A\eta_2,\dots,A\eta_n)=(\eta_1,\eta_2,\dots,\eta_n)
	\begin{pmatrix} \la_1 & \al\\ 0 & B\end{pmatrix}.\]
	化为
	\[
	P^* AP=\begin{pmatrix} \la_1 & \al\\ 0 & B\end{pmatrix}.
	\]
	由于$A$复正规,故$P^* AP$复正规,就有
	\begin{align*}
	\begin{pmatrix} \la_1 & \al\\ 0 & B\end{pmatrix}
	\begin{pmatrix} \overline{\la_1} & 0\\ \al^* & B^*\end{pmatrix}&=
		\begin{pmatrix} |\la_1|^2+|\al|^2 & \al B^*\\ B\al^* & BB^*\end{pmatrix},\\
	\begin{pmatrix} \overline{\la_1} & 0\\ \al^* & B^*\end{pmatrix}
	\begin{pmatrix} \la_1 & \al\\ 0 & B\end{pmatrix}&=
	\begin{pmatrix} |\la_1|^2 & \la_1\al\\ \overline{\la_1}\al^* & \al^*\al+B^* B\end{pmatrix}.
	\end{align*}
	比较可知$\al=0,\, BB^*=B^*B$. 显然$B$的所有特征复根为$\la_{2},\dots,\la_{n}$,
	故按假设$B$可以靠酉阵$P_1$准对角化为
	\[P_1^* B P_1=\diag(\la_2,\dots,\la_n)\]
	只要令$U=P\diag(1,P_1)$就结束战斗.
\end{proof}