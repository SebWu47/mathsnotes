\section{线性变换的Jordan化与Frobenius化}
从上一节的叙述来看,有限维线性空间上的线性变换$A$的对角化(或者等价的,线性空间可以写成$A$的所有特征空间的直和)的要求是苛刻的,一般情况下都达不到。

虽然从拓扑的角度来看,可以证明能对角化的线性变换(方阵)全体在$L(V)\;(M_n(F))$中是稠密的,以至于可以借助拓扑的办法来避免非对角化形式的纠缠,但那将留到后面来叙述。而且,拓扑的办法毕竟也不是万能的,故而这里还是需要从纯代数的角度来研究一般情况下的最简形式,也就是Jordan和Frobenius标准型。设$m_\al(x)$表示$\al$在相应的线性变换下的最小多项式。为了叙述的统一性,下面的引理是重要的。
\begin{lem}\label{lem:cover}
	设$F$是无限域,$V$是其上的线性空间,则对$V$的任意有限个真子空间$V_1,\dots,V_n$,都有
	\[V_1\cup V_2\cup \dots\cup V_n\neq V.\]
\end{lem}
利用这个定理,可以解决存在性问题或者不存在性问题(这两者可以相互转换).
\begin{thm}\label{thm:min}
	设$F$是无限域,$V$是其上的线性空间,$A$为$V$上的线性变换,若$A$有零化多项式,从而有最小多项式$m(x)$, 那么存在$\al\in V$,使得$m_\al (x)=m(x)$, 换句话说
	\[\deg m(x)=\max_{\beta\in V} \deg m_\beta(x).\]
	进一步的,若记$r=\deg m(x)$, 则对上面的那个$\al$有$\al,A\al,\dots,A^{r-1}\al$线性无关。
\end{thm}
\begin{proof}
	任取$\al\in V$,易知$m_\al(x)|m(x)$, 这说明当$\al$跑遍$V$时,$m_\al(x)$的总的个数是有限的,不妨将它们记为$m_1(x),m_2(x),\dots,m_k(x)$. 对每个$m_i(x)$, 考虑它能够零化的所有向量,
	\[V_i=\{\al\in V: m_i(A)\al=0\},\]
	则$V_i$为$V$的子空间且$V=V_1\cup \dots\cup V_k$. 

	由引理\ref{lem:cover},存在某个$V_j=V$, 即$m_j$能零化整个$V$, 从而$m|m_j$. 但显然的$m_j|m$,故而$m=m_j$. 从而存在某个$\al\in V$, 使得$m_\al(x)=m_j(x)=m(x)$. 现在若$\al,A\al,\dots,A^{r-1}\al$线性相关,则$\deg m_\al\leq r-1$,矛盾.
\end{proof}
注意到定理\ref{thm:min}中我们并没有要求$V$是有限维的,而只要求了$V$上的某个非零线性变换$A$有零化多项式。自然的,若$V$是有限维,不管$F$是否是无限域,$V$
上任何一个线性变换都有零化多项式,比如说特征多项式。但反过来要问,若$V$上存在某个非零线性变换$A$有零化多项式,能否得到$V$是有限维的,很不幸,这个猜想并不成立。

进一步的,是否存在某无限维线性空间$V$和其上线性变换$A$, 对任何$\al\in V$都有关于$A$的零化多项式,但是$A$自身没有零化多项式?或者, 存在某个$\al\in V$, 使得$\al$关于$A$就已经没有了零化多项式?下面的三个例子中的$D$均为普通求导运算。
\begin{exa}
	设$V=\{a+bx:a,b\in\ma{R}$, 则$V$可以看成$\ma{Q}$上的无限维线性空间,则
	\[D^2(f)=0,\;\forall f\in V.\]
\end{exa}
\begin{proof}
	后者显然,且显然的$D(x)=1$, 从而$D$非零,只需证$(V,\ma{Q})$是无限维的。为此考虑$V$的子空间$\ma{R}$, 若$\ma{R}$是有限维的,那么设
	\[r_1,r_2,\dots,r_n\]
	为一个基,就有$\ma{R}=\{k_1r_1+\dots+k_nr_n:k_i\in \ma{Q}\}$。注意到左边的势为连续势,右边与$\ma{Q}^n$等势,即为可数势,产生矛盾。
\end{proof}
\begin{exa}
	将$V=F[x]$看成$F$上的无限维线性空间,则对任何的$f\in F[x],\; \deg f=n$都有$D^nf=0$. 但是
	\[1+x+x^2+\dots+x^n+\cdots\]
	没有零化多项式。
\end{exa}
\begin{exa}
	考虑$D$作用在$C^{\infty} [0,1]$上,有
	\[\hat{D^n} \exp(x^2)=p_n'(x) \exp(x^2),\quad \forall n\geq 0,\]
	其中$p_n$为某个$n$次多项式。从此可知对任何的$0\neq f\in \ma{R}[x]$, $f(D) \exp(x^2)\neq 0$.
\end{exa}
这里还有一个问题,即若某线性空间$V$上的所有线性变换$A$都有零化多项式,能否得出$A$是有限维的?这个问题还有待研究。