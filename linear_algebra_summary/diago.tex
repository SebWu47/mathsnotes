\section{线性变换与相似对角化}
假设有一个域$F$上的$n$维线性空间$V$,$f$为上面的某一线性变换。如果仅要研究$V$本身,只需要任意找到一个基,就可以有分解
\[
    V=\Span{\alpha_1,\alpha_2,\dots,\alpha_n}\\
    =\Span{\alpha_1} \oplus \Span{\alpha_2}
    \oplus \dots \oplus \Span{\alpha_n},
\]
但这并没有多少意义,因为在每个小空间上$f$的性质并不清楚。因而我们要借助$f$来分解$V$,也就是要研究$(V,f)$这个共同体,或者更简洁的主要就是要研究$f$.

假设空间已有直和分解
\[V=W_1\oplus W_2\oplus\dots\oplus W_m,\]
为了清楚$f$在每个$W_i$上的行为,我们希望$W_i$均	为$f$的不变子空间,这个条件是必要的,因为这样才能让每个$f|W_i$仍然为$W_i$的线性变换,而不是其他什么东西。

不变子空间并不是随意构造出来的,然而幸运的是,$f$的任何一个特征子空间(如果存在)都是不变子空间,而且不同特征值的特征空间还一定恰好能够做直和。另外,在每个特征子空间里面,$f$限制在其上后的表示矩阵一定是对角化的,这样$f$在所有特征子空间的直和空间上也就有对角表示,形式十分简单。

为了达到这样的目的,就要考虑这个直和是否是$V$本身,因而总结出下面的一些引理或定理。下面用$p_F(\la),m_F(\la)$分别表示$f$在$F[\la]$中的特征与最小多项式,下标在不引起混乱时省略。用$V_{\la_i}$表示特征值$\la_i$所代表的特征子空间,在不致混淆时记为$V_i$
\begin{thm}\label{thm:vidrt}
    不论$f$是否可对角化,不论$p(\la),m(\la)$在$F[\la]$中是否能分解成一次式的乘积,若$f$有一些不同的特征值$\la_1,\la_2,\dots,\la_m\in F$,其中$\la_i\neq \la_j\;(i\neq j)$,那么
    \[V_1+V_2+\dots +V_m=V_1\oplus V_2\oplus \dots \oplus V_m,\]
    即若$\{\al_1^i,\al_2^i,\dots,\al_{r_i}^i\}\;(1\leq i\leq m)$分别为$V_i$的线性无关组,则他们的并集仍线性无关。这说明$f$可对角化等价于
    \begin{equation}
    V=V_1\oplus V_2\oplus\dots\oplus V_m.\label{drtbreak}
    \end{equation}
    其中$\la_1,\dots,\la_m$为$f$所有不同特征值。
\end{thm}
\begin{proof}
    假设$0=\al_1+\dots+\al_k$,其中$k\leq m, \al_i\in V_i$. 那么用$f$作用两边得$0=\la_1\la_1+\dots+\la_k\al_k$,从而有
    \[(\la_1-\la_2)\al_2+\dots+(\la_1-\la_k)\al_k=0.\]
    假设$k-1$的时候已经为直和分解,则有$\al_2=\dots\al_k=0$, 从而最终$\al_1=0$,故为直和。而线性无关的叙述是直和的等价定义方式。
    
    若\eqref{drtbreak}成立, 则任取$V_i$的基,将他们合成$V$的基,$f$在这个基下的矩阵一定是
    $\diag(\la_1,\dots,\la_m)$其中$\la_i$出现$\dim V_i$次。反过来,若$f$可对角化,则可假设
    \begin{align*}
    f(\al_1^1,\al_2^1,&\dots,\al_{r_1}^1,\dots,\al_1^m,\al_2^m,\dots,\al_{r_m}^m)\\
    &=(\al_1^1,\al_2^1,\dots,\al_{r_1}^1,\dots,\al_1^m,\al_2^m,\dots,\al_{r_m}^m)\diag(\la_1,\dots,\la_m)
    \end{align*}
    其中括号中为$V$的基,因而$r_1+\dots+r_m=n$。注意到显然有
    $\{\al_i^1,\dots,\al_{r_i}^i\}$为$V_i$中的线性无关组,从而$r_1\leq \dim V_i$, 故
    \[n=r_1+\dots+r_n\leq \dim V_1+\dots +\dim V_m\leq n,\]
    立刻得到\eqref{drtbreak}.
\end{proof}

\begin{thm}\label{thm:pla}
    不论$f$是否可对角化,假设$p(\la)$在$F[\la]$中有分解
    \begin{equation}
    	p(\la)=(\la-\la_1)^{r_1} \dots (\la-\la_m)^{r_m} g(\la),\label{pla}
    \end{equation}
    其中$\la_i\in F$且两两不同,$r_i\geq 1$且$g(\la)$在$F[\la]$中无一次因式,那么$\la_1,\dots,\la_m$为$f$在$F$中所有的特征值且
    \[1\leq \dim V_i\leq r_i.\]
    进一步的,$f$在$(V,F)$上可对角化等价于$g(\la)=1$且$\dim V_i=r_i$. 并且这个条件是最优的,即$g(\la)=1$和$\dim V_i=r_i$相互独立,谁也不蕴含对方。
\end{thm}
\begin{proof}
	设$(\al_1,\dots,\al_n)$为$V$的一个基,$f$在其上的矩阵为$A$, 那么
	\begin{align*}
		&\la_0 \mbox{为} f \mbox{在} F \mbox{中的特征值}\\
		&\lefr\la_0\in F, \exists 0\neq \al \in V, f(\al)=\la_0 \al\\
		&\lefr\la_0\in F, \exists 0\neq x \in F^n, f(\al_1,\dots,\al_n)x=\la_0 (\al_1,\dots,\al_n)x\\
		&\lefr\la_0\in F, \exists 0\neq x\in F^n, Ax=\la_0 x\\
		&\lefr\la_0\in F, p(\la_0)=\det(\la_0 I-A)=0.
	\end{align*}
	在注意到$g(\la)$在$F[\la]$中无一次因式,说明$\la_1,\dots,\la_m$为$f$在$F$中的全部特征值,因而$\dim V_i\geq 1$.

	现在假设$s_i=\dim V_i$, 取$\{\al_1^i,\al_2^i,\dots,\al_{s_i}^i\}\;(1\leq i\leq m)$分别为$V_i$的基,将他们扩展为$V$的一个基
	\[[\al]:=\al_1^1,\dots,\al_{s_1}^1,\dots,\al_m^1,\dots,\al_{s_m}^m,\beta_1,\dots,\beta_t\]
	其中$s_1+\dots+s_m+t=n,\;t\geq 0$. 则$f$在这个基上有
	\[f([\al])=[\al]
	\begin{pmatrix}
		\diag(\la_1,\dots,\la_m) & B\\
		0 & C
	\end{pmatrix},\]
	其中$B\in F^{(n-t)\times t)},\; C\in F^{t\times t}$且$\diag(\la_1,\dots,\la_m)$中$\la_i$出现$s_i$次。从而有
	\[p(\la)=(\la-\la_1)^{s_1} \dots (\la-\la_m)^{s_m} \det(\la I_t-C),\]
	与\eqref{pla}比较,由于$g$在$F[\la]$中已无一次因式,而$\det(\la I_t-C)$可能还有,因而$s_i\leq r_i$.

	假设$f$在$F$中可对角化,由定理\ref{thm:vidrt},有$s_1+\dots+s_m=n$,从而
	\[n=s_1+\dots+s_m\leq r_1+\dots +r_m\leq r_1+\dots+r_m+\deg g=n.\]
	立即得到$r_i=s_i$ 并且$\deg g=0$, 也就是$g=1$. 反过来由于$g=1$,有$r_1+\dots+r_m=n$, 从而
	\[s_1+\dots+s_m=r_1+\dots+r_m=n,\]
	再次利用定理\ref{thm:vidrt}, $f$就可对角化。

	最后来说明两个条件的独立性.
	\begin{description}
		\item[(a)] $F=\mathbb{C}$, 考虑矩阵$A=I_2+E_{21}$, $p(\la)=(\la-1)^2$, 这说明$g(\la)=1$, 但是
		\[\dim V_1=2-\rank(I_2-A)=2-\rank(E_{21})=1<2,\]
		此时$A$不可对角化.
		\item[(b)] $F=\mathbb{R}$, 考虑矩阵
		\[B=\begin{pmatrix}
		  1 & &\\
		  &0&-1\\
		  &1&0\end{pmatrix},\]
		有$p(\la)=(\la-1)(\la^2+1)\in \mathbb{R}[\la]$, 有$g(\la)=\la^2+1$. 由于$1\leq \dim V_1=1$, 知的确有
		$\dim V_1$恰为$\la-1$在$p(\la)$中的次数,此时$B$不可对角化。
	\end{description}
\end{proof}
注意定理\ref{thm:pla}中最重要的是$p(\la)$恰好包含$f$在$F$中的全部特征值,正是这一特性使得我们将$p(\la)$称为$f$的特征多项式,也有了通过它判别所有特征子空间的和是否是原空间的可能性,也就是对角化$f$的可能性. 但这就衍生出三个问题:

一是随意给出$f$在$F[\la]$中的一个零化多项式,它是否包含$f$在$F$中的全部特征值. 这个问题的答案是肯定的,这由我们下面关于最小多项式的定理保证。

二是既然任意一个零化多项式包含全部特征值,那么它能不能通过它来判定是否能对角化$f$。很不幸,这一问题的答案是否定的。观察$p(\la)$的定义与定理$\ref{thm:pla}$的证明过程可以知道,除了包含全部特征值之外,$p(\la)$还有一个性质,那就是
\[r_1+\dots+r_m+\deg g=n,\quad r_i\geq \dim V_i\geq 1.\]
正是这一性质使得$p(\la)$可以将$V_i$的直和与$V$联系起来,也就使得$p(\la)$最终可以用来判定$f$的对角化。但是,这个条件是可以修改的,也就是利用最小多项式来判定。

三是对于一般的零化多项式,虽然不能用来判定$f$的对角化,那它能否反映出来$f$的某些次一级的特性。这个问题的答案是肯定的(它的答案就是空间的准素分解),而它能成功的原因就在于对于任何一个$h(\la)\in F[\la]$, $\ker h(f)$均为$f$的不变子空间。

为了空间分解的方便,准素分解引理是重要的.
\begin{lem}\label{lem:prime}
	对任何的$g(\la)\in F[\la]$与它在$F[\la]$中的分解
	\[g(\la)=g_1(\la)g_2(\la)\dots g_m(\la),\]
	其中$\gcd(g_1,\dots,g_m)=1$, 那么
	\[\ker g(f)=\ker g_1(f)\oplus \ker g_2(f)\oplus\dots\oplus\ker g_m(f).\]
\end{lem}
\begin{thm}
	不论$f$是否可对角化,假设$p(\la),m(\la)$在$F[\la]$中有分解\
	\begin{align*}
    	m(\la)=(\la-\la_1)^{r_1} \dots (\la-\la_m)^{r_m} g(\la),
    \end{align*}
    其中$\la_i\in F$且两两不同,$r_i\geq 1$且$g(\la)$在$F[\la]$中无一次因式,那么$\la_1,\dots,\la_m$为$f$在$F$中所有的特征值. 注意此时$r_i$与$\dim V_i$的大小关系无法确定。

    进一步的,$f$在$(V,F)$上可对角化等价于
    \[g(\la)=1\quad\mbox{且}\quad r_1=r_2=\dots=r_m=1.\]
    并且这个条件是最优的,即$g(\la)=1$和$r_1=\dots=r_m=1$相互独立,谁也不蕴含对方。
\end{thm}
\begin{proof}
	显然$p(\la_i)=m(\la_i)=0$, 因而$\la_i$均为$f$特征值。反过来,假设$\la_0\in F$为$f$特征值,那么存在$0\neq \al\in V$,使得$f(\al)=\la_0\al$. 容易知道$m(f)\al=m(\la_0)\al=0$, 因而$m(\la_0)=0$。 故而$\la_1,\dots,\la_m$的确为$f$在$F$中的全部特征值。

	现在假设$f$可对角化,即\eqref{drtbreak}成立,任取$\al\in V$, 做分解
	\[\al=\al_1+\al_2+\dots+\al_m,\quad f(\al_i)=\la_i \al_i.\]
	再试着用全一次的多项式来作用它,即记$h(\la)=(\la-\la_1)\dots(\la-\la_m)$, 有
	\begin{align*}
		h(f)\al_i&=\prod_{j\neq i}(f-\la_j)(f-\la_i)\al\\
		&=\prod_{j\neq i}(f-\la_j)0\\
		&=0,
	\end{align*}
	从而$h(f)\al=0$. 由于$m(\la)$为最小多项式,故$m(\la)|h(\la)$. 又由于$m(\la)$包含$f$在$F$中的全部特征值,因而就有$m(\la)=h(\la)$. 反过来,假设$m(\la)=h(\la)$,利用引理\ref{lem:prime}立知$V$可分解为全部特征子空间的直和,因而$f$可对角化.

	最后来说明$g(\la)=1$与$r_1=\dots=r_m=1$的独立性, $r_1$与$\dim V_i$的关系, 依然使用定理\ref{thm:pla}中类似的例子,
	\begin{description}
		\item[(a)] $F=\mathbb{C}$, 考虑矩阵$A=I_2+E_{21}$, $p(\la)=(\la-1)^2$. 由于$A-I\neq 0$, 因而$\la-1$不是零化多项式,从而$m(\la)=(\la-1)^2$, 此时$A$不可对角化。此时$\dim V_1=1<2$.
		\item[(b)] $F=\mathbb{R}$, 考虑矩阵
		\[B_1=\begin{pmatrix}
			0 & -1\\
			1 & 0
		  \end{pmatrix},\quad
		  B=\diag (1,B_1,B_1).\]
		  对于$B$有$p(\la)=(\la-1)(\la^2+1)^2\in \mathbb{R}[\la]$, 因而$B$不可对角化,从而$m(\la)\neq \la-1$. 另一方面,容易验证$(\la-1)(\la^2+1)$已经是零化多项式,从而$m(\la)=(\la-1)(\la^2+1)$. 从这个例子还可以看出,最小多项式的不为一次的不可约式与特征多项式的不一定相等。
		\item[(c)] 考虑$I_2$, 显然$m(\la)=\la-1$, $\dim V_1=2-\rank(I_2-I_2)=2$.
	\end{description}
\end{proof}

从上面可以看出特征与最小多项式用起来十分方便,然而注意到,一个多项式在一个域与它的扩域中的最终分解形式不一定相等,也就使得在大域中可以对角化的矩阵不一定在小域中也能对角化,例如定理\ref{thm:pla}中例子(b)中的$B$, 容易知道$p_{\ma{R}}(\la)=(\la-1)(\la^2+1)$, 从而$B$不可实对角化,但是$p_{\ma{C}}(\la)=(\la-1)(\la+i)(\la-i)$, 所以$B$可复对角化为$\diag(1,i,-i)$. 当然,幸运的是$p_{\ma{R}}(\la)=p_{\ma{C}}(\la)$, 这就使得我们无需计算两次$p(\la)$(这由$p(\la)$的定义显然得到). 更幸运的是,最小多项式也有同样的性质,即虽然分解式不同,但大小域上的最小多项式本身相同.
