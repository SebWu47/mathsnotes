\chapter{Measure Theory}
\section{Rings and Algebras}
\begin{pro}%1
Prove that
	\[(\liminf E_n)^c=\limsup E_n^c,\,(\limsup E_n)^c=\liminf E_n^c.\]
\end{pro}
\begin{proof}
	We only prove the first one. If $x$ is a member of the left set, then there must be infinitely many $n$, such that $x$ is not the member of those $E_n$, which means $x$ belongs to the right set. 
\end{proof}

\begin{pro}%2
	Prove that
	\[\limsup E_n=\bigcap_{k=1}^\infty \bigcup_{n=k}^\infty E_n,\,\liminf E_n=\bigcup_{k=1}^\infty \bigcap_{n=k}^\infty E_n.\]
\end{pro}
\begin{proof}
	We prove the first one. If $x$ belongs to the left, then for any $k$, there is a $n(k)\geq k$, such that $x\in E_{n(k)}$, i.e.
	\[x\in\bigcup_{n=k}^\infty E_n.\]
	Since the $k$ can be any number, we know $x$ belongs to the right. And the vise versa is similar.
\end{proof}

\begin{pro}%3
	If $\ma{R}$ is a $\sigma$-ring and $E_n\in\ma{R}$, then
	\[\bigcap_{n=1}^\infty E_n\in\ma{R},\,\limsup E_n\in\ma{R},\liminf E_n\in\ma{R}.\]
\end{pro}
\begin{proof}
	For any $k\geq 1$, let $F_k=\bigcap_{n=1}^k E_n,$ then
	\[\bigcap_n E_n=\bigcap_k F_k,\, F_{k+1}\subset F_{k}\in\ma{R}.\]
	Hence
	\begin{align*}
	\bigcap_n E_n&= F_1-\big(F_1-\bigcap_k F_k\big)\\
				 &= F_1-\bigcup_k (F_1-F_k)\in\ma{R}.
	\end{align*}
	And the last two are easy to be shown by the first one and the problem right above.
\end{proof}

\begin{pro}%4
	The intersection of any collection of rings(algebras,$\sigma$-rings or $\sigma$-algebras) is also a ring (algebra,$\sigma$-ring or $\sigma$-algebra)
\end{pro}
\begin{proof}
	Easy to get from the definition of Intersection.
\end{proof}

\begin{pro}%5
	If $\ma{D}$ is any class of sets, then there exists a unique ring $\ma{R}_0$ such that (i) $\ma{R}_0\supset \ma{D}$, and (ii) any ring $\ma{R}$ containing $\ma{D}$ contains also $\ma{R}_0$. $\ma{R}_0$ is called the ring generated by $\ma{D}$, and is denoted by $\ma{R}(\ma{D})$.
\end{pro}
\begin{proof}
	Let $J$ be the set consisting of the subsets of any set in $\ma{D}$ i.e.
	\[J=\bigcup_{A\in\ma{D}}P(A),\]
	where $P(A)$ is the power set of $A$. Then let $\ma{R}$ be the set of the union of finitely many sets in $J$, i.e. 
	\[\ma{R}=\{A\colon \exists\;n\geq 1, B_1,\dots,B_n\in J,\,\mbox{such that} \,A=\bigcup_{i=1}^n B_i\}.\]
	Easy to show that $\ma{R}$ is a ring containing $\ma{D}$. Now define
	\[\ma{R}_0=\bigcap_{\alpha \in I} \ma{R}_{\alpha},\, I=\{\alpha\colon \ma{R}_{\alpha}\,\mbox{is a ring containing}\,\ma{D}\},\]
	then $\ma{R}_0$ (i) is also a ring containing $\ma{D}$ by the problem right above, and (ii) is a subset of any ring containing $\ma{D}$ due to the definition of $\ma{R}_0$ itself. Now consider there is another ring satisfying (i) and (ii), say $\ma{R}^*$, then $\ma{R}_0\subset \ma{R}^*$ by definition of $\ma{R}_0$ and $\ma{R}^*\subset \ma{R}_0$ by definition of $\ma{R}^*$. Thus we prove the uniqueness.
\end{proof}

\begin{pro}%6
	If $\ma{D}$ is any class of sets, then there exists a unique $\sigma$-ring $\ma{P}_0$ such that (i) $\ma{P}_0\supset\ma{D}$, and (ii) any $\sigma$-ring $\ma{P}$ containing $\ma{D}$ contains also $\ma{P}_0$. $\ma{P}_0$ is called the $\sigma$-ring generated by $\ma{D}$, and is denoted by $\ma{P}(\ma{D})$. A similar result holds for $\sigma$-algebra.
\end{pro}
\begin{proof}
	Let $J$ be the set consisting of the subsets of any set in $\ma{D}$ i.e.
	\[J=\bigcup_{A\in\ma{D}}P(A),\]
	where $P(A)$ is the power set of $A$. Then let $\ma{P}$ be the set of the union of infinitely many sets in $J$, i.e. 
	\[\ma{P}=\{A\colon B_1,\dots,B_n,\ldots\in J,\,\mbox{such that} \,A=\bigcup_{n=1}^\infty B_n\}.\]
	Easy to show that $\ma{P}$ is a $\sigma$-ring containing $\ma{D}$. Now define
	\[\ma{P}_0=\bigcap_{\alpha \in I} \ma{P}_{\alpha},\, I=\{\alpha\colon \ma{P}_{\alpha}\,\mbox{is a $\sigma$-ring containing}\,\ma{D}\},\]
	then $\ma{P}_0$ (i) is also a $\sigma$-ring containing $\ma{D}$ by the problem $1.1.4$, and (ii) is a subset of any $\sigma$-ring containing $\ma{D}$ due to the definition of $\ma{P}_0$ itself. Now consider there is another $\sigma$-ring satisfying (i) and (ii), say $\ma{P}^*$, then $\ma{P}_0\subset \ma{P}^*$ by definition of $\ma{P}_0$ and $\ma{P}^*\subset \ma{P}_0$ by definition of $\ma{P}^*$. Thus we prove the uniqueness.
\end{proof}

\begin{pro}%7
	If $\ma{D}$ is any class of sets, then every set in $\ma{R}(\ma{D})$ can be covered by a finite union of sets of $\ma{D}$.
\end{pro}
\begin{proof}
	Recall what we do in Problem $1.1.5$, we know that $\ma{R}$ defined by $J$ is a ring containing $\ma{R}(\ma{D})$, then  we can get the result by the definition of $\ma{R}$ itself.
\end{proof}

\begin{pro}%8
	If $\ma{D}$ is any class of sets, then every set in $\ma{P}(\ma{D})$ can be covered by a finite union of sets of $\ma{D}$.
\end{pro}
\begin{proof}
	A similar result holds when we notice $\ma{P}$ defined by $J$ is a $\sigma$-ring containing $\ma{P}(\ma{D})$.
\end{proof}

\begin{pro}%9
	Let $\ma{D}$ consist of those sets which are either finite or have a finite complement. Then $\ma{D}$ is an algebra. If $X$ is not finite, then $\ma{D}$ is not a $\sigma$-algebra.
\end{pro}
\begin{proof}
	Given any $A,B$ in $\ma{D}$, notice that $A-B\subset A,B^c$, so $A-B\in\ma{D}$ when $A$ or $B^c$ is finite. When both $A^c$ and $B$ are finite, the result also holds since $|(A-B)^c|=|A^c|+|A\cap B|$. 

	As for $A\cup B$, if $A^c$ or $B^c$ is finite, then $A\cup B$ is in $\ma{D}$ since $A^c\cap B^c\subset A^c,B^c$. The result also holds when both $A$ and $B$ are finite by $A\cup B=A\cup(A-B)$. So we proved the first part, as for the second part, just think $X$ as $\m{N}$ or $\m{R}$.
\end{proof}

\begin{pro}%10
	Let $\ma{K}$ consist of all sets of the form $\cup_{k=1}^n J_k$, where the $J_k$ are mutually disjoint intervals on the real line, having the form $\{t\colon a<t\leq b\}$, where $a$ and $b$ are any real numbers. Then $\ma{K}$ is a ring but not an algebra.
\end{pro}
\begin{proof}
	Notice that \[(a,b]\cup (c,d]=(\min\{a,c\},\max\{b,c\}]\] when $(a,b]\cap (c,d]\neq\emptyset$ and $(a,b]\ba (c,d]=(a,c]\cup (d,b]$, so $\ma{D}$ is a ring. Suppose we have finitely many intervals like $(a,b]$, then we must get the maximum one of those $b$s, which is the maximum number in the union, but $\m{R}$ doesn't have  a maximum member, which leads to a contradiction.
\end{proof}

\begin{pro}%11
	Find $\ma{R}(\ma{D})$ when $\ma{D}$ consists of two distinct sets, each consisting of one point.
\end{pro}
\begin{proof}
	It's like $\ma{R}=\{\emptyset,\{x\},\{y\},\{x,y\}\}$.
\end{proof}

\section{Definition of Measure}
\begin{pro}%1
	If $\mu$ satisfies the properties (i)-(iii) in Definition $1.2.1$, and if $\mu(E)<\infty$ for at least one set $E$, then (iv) is also satisfied.
\end{pro}
\begin{proof}
	Since $\mu(E)=\mu(E)+\sum_{n=1}^\infty \mu(\emptyset)$ where $\mu(E)<\infty$, we have $\mu(\emptyset)=0$.
\end{proof}

\begin{pro}%2
	Let $X$ be an infinite space. Let $\ma{K}$ be the class of all subsets of $X$. Define $\mu(E)=0$ if $E$ is finite and $\mu(E)=\infty$ if $E$ is infinite. Then $\mu$ is finitely additive but not completely additive.
\end{pro}
\begin{proof}
	Given any disjoint subsets of $X$, say $A,B$, then
	\[\mu(A\cup B)=
	\begin{cases}
	\infty,& \mbox{if}\,A\,\mbox{or}\,B\,\mbox{is infinite},\\
	0,&\mbox{if}\,A\,\mbox{and}\,B\,\mbox{are finite}.
	\end{cases}
	\]
	which leads to the first part. Now let $A_n=\{x_n\}$ where $x_n\neq x_m$ when $n\neq m$, and 
	\[\mu(\sum_{n=1}^\infty A_n)=\infty,\quad\sum_{n=1}^\infty \mu(A_n)=0.\]
\end{proof}

\begin{pro}%3
	If $\mu$ is a measure on a $\sigma$-algebra $\ma{K}$, and if $E,F$ are sets of $\ma{K}$, then
	\[\mu(E)+\mu(F)=\mu(E\cup F)+\mu(E\cap F).\]
\end{pro}
\begin{proof}
	Since $F=(F\cap E)\cup (F\ba E)$, we have
	\begin{align*}
	\mu(E)+\mu(F)&=\mu(E)+\mu(F-E)+\mu(E\cap F)\\
				 &=\mu(E\cup F)+\mu(E\cap F).
	\end{align*}
\end{proof}

\begin{pro}%4
	Let $\{\mu_n\}$ be a sequence (finite or infinite) of measures defined on the same $\sigma$-algebra $\ma{K}$. Define $\sum_{n=1}^\infty \mu_n$ by \[\bigg(\sum_{n=1}^\infty \mu_n\bigg) (E)=\sum_{n=1}^\infty \mu_n(E)\] for every $E\in\ma{K}$. Prove that $\sum_{n=1}^\infty \mu_n$ is a measure.
\end{pro}
\begin{proof}
	We only show it is completely additive, before that we prove a lemma first.
	\begin{lem}
		Suppose $a_{nm}\geq 0$ for every $n,m\geq 1$, then 
		\[\sum_n\sum_m a_{nm}=\sum_m \sum_n a_{nm},\]
		where the result could be $\infty$ sometimes.
	\end{lem}
	\begin{proof}
		Suppose $\sum_{m}\sum_n a_{nm}$ converges, then $\sum_n a_{nm}<\infty$ for every $m$, thus
		\[\sum_{i=1}^n\sum_{m} a_{im}=\sum_m\bigg(\sum_{i=1}^n a_{im}\bigg)\leq \sum_m\sum_n a_{nm}<\infty,\]
		thus $\sum_{n}\sum_{m} a_{nm}$ converges and
		\[\sum_{n}\sum_{m} a_{nm}\leq \sum_{m}\sum_{n} a_{nm}.\]
		Now completely similar we have
		\[\sum_{m}\sum_{n} a_{nm}\leq \sum_{n}\sum_{m} a_{nm}.\]
		which leads to the result. As for $\sum_{m}\sum_{n} a_{nm}$ diverges, we can show $\sum_{n}\sum_{m} a_{nm}$ also diverges by proof of contradiction.
	\end{proof}
	Now we focus on the original question, given any mutually disjoint measurable sets $\{E_m\}_{m=1}^\infty$, we have
	\begin{align*}
		\bigg(\sum_n \mu_n \bigg)\bigg(\bigcup_m E_m\bigg)&=\sum_n \mu_n\bigg(\bigcup_m E_m\bigg)\\
											 &=\sum_n \sum_m \mu_n(E_m)\\
											 &=\sum_m \sum_n \mu_n(E_m)\\
											 &=\sum_m \bigg(\sum_n \mu_n \bigg)(E_m).
	\end{align*}
	So we finally finish our proof. 
\end{proof}

\begin{pro}%5
	Let $X$ consist of a sequence $\{x_m\}$ and let $\{p_m\}$ be a sequence of nonnegative numbers. For any subset $A\subset X$, let
	\[\mu(A)=\sum_{x_m\in A} p_m.\]
	Then $\mu$ is a $\sigma$-finite measure.
\end{pro}
\begin{proof}
	Given any mutually disjoint measurable sets $\{A_n\}$, we have
	\begin{align*}
		\mu(\cup_n A_n)&= \sum_{\exists\; n,x_m\in A_n} p_m\\
					   &= \sum_n\bigg(\sum_{x_n\in A_n} p_m\bigg)\\
					   &= \sum_n \mu(A_n),
	\end{align*}
	since every $p_m$ is nonnegative. Also, $\mu(X)=\sum_{x_m\in X} p_m$ and such $x_m\in X$ are countable at most, so we know $\mu$ is $\sigma$-finite.
\end{proof}

\begin{pro}%6
	Given an example of a measure $\mu$ and a monotone-decreasing sequence $\{E_n\}$ of $\ma{K}$ such that $\mu(E_n)=\infty$ for all $n$, and $\mu(\lim_n E_n)=0$.
\end{pro}
\begin{proof}
	Let $\mu(E)=0$ if $E=\emptyset$ and $\mu(E)=\infty$ if $E\neq\emptyset$, then the conclusion is trivial.
\end{proof}

\section{Outer Measure}
\begin{pro}%1
	Define $\mu^*(E)$ as the number of points in $E$ if $E$ is finite and $\mu^*(E)=\infty$ if $E$ is infinite. Show that $\mu^*$ is an outer measure. Determine the measurable sets.
\end{pro}
\begin{proof}
	Notice that 
	\begin{align*}
		\mu^*\bigg(\bigcup_n E_n\bigg)&=\sum_n \bigg(E_n\ba\bigcup_{i=1}^{n-1} E_i\bigg)\\
		&\leq \sum_n \mu^*(E_n).
	\end{align*}And the measurable set is any subset of $X$.
\end{proof}

\begin{pro}%2
	Define $\mu^*(\emptyset)=0$ and $\mu^*(E)=1$ if $E\neq\emptyset$. Show that $\mu^*$ is an outer measure, and determine the measurable sets.
\end{pro}
\begin{proof}
	Outer is trivial and the m-set is only empty set.
\end{proof}

\begin{pro}%3
	Let $X$ have a non-countable number of points. Set $\mu^*(E)=0$ if $E$ is countable at most and $\mu^*(E)=1$ if $E$ is non-countable. Show that $\mu^*$ is an outer measure, and determine the measurable sets.
\end{pro}
\begin{proof}
	Outer is right since \emph{countable union of countable sets is also countable} and the m-sets are sets countable at most.
\end{proof}

\begin{pro}%4
	If $\mu^*$ is an outer measure and $B$ is a fixed set, then the set function $\nu^*$ given by $\nu^*(A)=\mu^*(A\cap B)$ is also an outer measure. Find the relation between the measurable sets of $\mu^*$ and $\nu^*$.
\end{pro}
\begin{proof}
	Easy to show that $\nu^*$ is outer and the relation is every $\mu^*$-m-set is $\nu^*$-m-set.
\end{proof}

\begin{pro}%5
	If $\{\mu_n^*\}$ is a sequence of outer measures, then $\sum_n \mu_n^*$ defined similar to problem $1.2.4$ is also a outer measure.
\end{pro}
\begin{proof}
	Completely similar to problem $1.2.4$
\end{proof}

\begin{pro}%6
	Prove that if an outer measure is finitely additive, then it is a measure.
\end{pro}
\begin{proof}
	Given any mutually disjoint sets $\{E_n\}$, we have
	\[\mu^*\bigg(\bigcup_n E_n\bigg)\geq \mu^*\bigg(\bigcup_{i=1}^n E_i\bigg)=\sum_{i=1}^n \mu^*(E_i).\]
	Then by taking limit both sides, we have
	\[\mu^*\bigg(\bigcup_n E_n\bigg)\geq\sum_n \mu^*(E_n).\]
\end{proof}

\section{Construction of Outer Measures}
\begin{pro}%1
	Let $\ma{K}$ consist of $X,\emptyset$, and all the one-point sets. Let $\lambda(X)=\infty$ and $\lambda(\emptyset)=0,\lambda(E)=1$ if $E\neq X, E\neq \emptyset$. Describe the outer measure.
\end{pro}
\begin{proof}
	The outer measure is exactly the number of points in a set.
\end{proof}

\begin{pro}%2
	Let $X$ be a non-countable space and $\ma{K}$ be as in Problem $1.4.1$. Let $\lambda(X)=1,\lambda(E)=0$ if $E\neq X$. Describe the outer measure.
\end{pro}
\begin{proof}
	The outer measure is always $0$ for any set.
\end{proof}

\begin{pro}%3
	Show that, under the conditions of Theorem $1.4.1$, $\mu^*(E)\leq \lambda(E)$ if $E\in \ma{K}$,. Give an example where inequality holds.
\end{pro}
\begin{proof}
	The outer measure in Problem $1.4.2$ just holds the inequality.
\end{proof}

\begin{pro}%4
	If $\ma{K}$ is a $\sigma$-algebra and $\lambda$ is a measure on $\ma{K}$, then $\mu^*(A)=\lambda(A)$ for any $A\in\ma{K}$.
\end{pro}
\begin{proof}
	Since $\lambda(E)\geq \lambda(A)$ for any $E\in\ma{K},E\supset A$, thus we have 
	\begin{align*}
	\mu^*(A)&=\inf\{\sum_{n}\lambda(E_n)\colon E_n\in\ma{K},\bigcup_n E_n\supset A\}\\
			&\geq \inf\{\lambda(\bigcup_n E_n)\colon E_n\in\ma{K},\bigcup_n E_n\supset A\}\\	
	        &\geq \lambda(A),
	 \end{align*}
	since $\lambda$ is a measure on $\ma{K}$. And the inverse $\mu^*(A)\leq \lambda(A)$ also holds by $A\in\ma{K}$.
\end{proof}

\begin{pro}%5
	If $\ma{K}$ is a $\sigma$-algebra and $\lambda$ is a measure on $\ma{K}$, then every set in $\ma{K}$ is $\mu^*$-measurable.
\end{pro}
\begin{proof}
	For any $E\in\ma{K},A\subset X$ and $\epsilon>0$, there exists a sequence $\{E_n\}$, such that $E_n\in \ma{K}$ and $\cup_n E_n\supset A$ and
	\begin{align*}
		\mu^*(A)+\epsilon&\geq \lambda(\cup_n E_n)\\
						 &=\lambda(\cup_n E_n\cap E)+\lambda(\cup_n E_n\ba E)\\
						 &\geq \mu^*(A\cap E)+\mu^*(A\ba E).
	\end{align*}
	Thus the conclusion holds.
\end{proof}

\section{Completion of Measures}
\begin{pro}%1
	Let $\mu$ be a complete measure. A set for which $\mu(N)=0$ is called a null set. Show that the class of null sets is a $\sigma$-ring. Is it also a $\sigma$-algebra?
\end{pro}
\begin{proof}
	It is a $\sigma$-algebra if and only if $\mu(X)=0$.
\end{proof}

\begin{pro}%2
	Let the conditions of Theorem $1.5.1$ hold and denote by $\underline{\ma{K}}$ the class of all sets of the form $E-N$ where $E\in \ma{K}$ and $N$ is any subset of a set of $\ma{K}$ having measure zero. Then \[(X,\overline{\ma{K}},\overline{\mu})=(X,\underline{\ma{K}},\underline{\mu})\] where $\underline{\mu}$ is defined by $\underline{\mu}(E\ba N)=\mu(E)$.
\end{pro}
\begin{proof}
	First we rewrite the definition of $\overline{\ma{K}}$ and $\underline{\ma{K}}$ and keep all symbol in use.
	\begin{align*}
		\overline{\ma{K}}&=\{E\cup N\colon E\in\ma{K},N\subset M\in \ma{K},\mu(M)=0\},\\
		\underline{\ma{K}}&=\{E\ba N\colon E\in\ma{K},N\subset M\in \ma{K},\mu(M)=0\}.
	\end{align*}
	Since 
	\[E\cup N=(E\cup M)\ba(M\ba N),\]
	and
	\[E\ba N=(E\ba M)\cup(M\ba N),\]
	we know $\overline{\ma{K}}=\underline{\ma{K}}$. Easy to know $\underline{\mu}(E\ba N)=\mu(E)=\overline{\mu}(E\cup N)$, so $\underline{\mu}$ is well defined by the well definition of $\overline{\mu}$. Also we notice that
	\begin{align*}
		\underline{\mu}(E\cup N)&=\underline{\mu}\big((E\cup M)\ba (M\ba N)\big)\\
								&=\mu(E\cup M)\\
								&=\mu(E)\\
								&=\overline{\mu}(E\cup N),
	\end{align*}
	so $\overline{\mu}=\underline{\mu}$ on $\overline{\ma{K}}=\underline{\ma{K}}$.
\end{proof}

\section{The Lebesgue And The Lebesgue-Stieltjes Measures}
\begin{pro}%1
	The Lebesgue measure of a point set is zero.
\end{pro}
\begin{proof}
	For any $x\in\m{R}^n$ and $\epsilon>0$, we define $(x-\epsilon,x+\epsilon)\in\ma{K}$ by
	\[(x-\epsilon,x+\epsilon)=\{(y_1,\dots,y_n)\colon x_i-\epsilon<y_i<x_i+\epsilon,1\leq i\leq n\}.\]
	Then 
	\begin{align*}
		\mu^*(\{x\})&\leq \mu^*(x-\epsilon,x+\epsilon)\\
		&\leq \lambda(x-\epsilon,x+\epsilon)\\
		&=(2\epsilon)^n,
	\end{align*}
	which leads to that $\mu^*(\{x\})=0$. Now we prove that $\{x\}$ is a Lebesgue set, in other words, is Lebesgue (outer) measurable.
	Since
	\begin{align*}
		\mu^*(A)&\leq \mu^*(A\cap\{x\})+\mu^*(A\ba \{x\})\\
		&=\mu^*(A\ba \{x\})\\
		&\leq \mu^*(A),
	\end{align*}
	we conclude that $\{x\}$ is indeed a Lebesgue set.
\end{proof}

\begin{pro}%2
	The Lebesgue measure of a countable set of points is zero.
\end{pro}
\begin{proof}
	It's easy to know that a countable set is a Lebesgue set by the set of a single point is a Lebesgue set.
	And hence the measure is $0$ since a set of a single point is already $0$.
\end{proof}

\begin{pro}%3
	The outer Lebesgue measure of a closed bounded interval $[a,b]$ on the real line is equal to $b-a$.
\end{pro}
\begin{proof}
	Consider any covering of $[a,b]$  by $\cup_{n} (a_n,b_n)\supset [a,b]$, by the Theorem of Heine-Borel, there is a finite covering 
	\[\bigcup_{i=1}^{m}(a_i,b_i)=(\min\{a_i\},\max\{b_i\}).\]
	Thus 
	\begin{align*}
		\mu^*([a,b])&=\inf\{\sum_{n} b_n-a_n\colon \bigcup_{n}(a_n,b_n)\supset [a,b]\}\\
		 			&\geq \inf\{\max\{b_i\}-\min\{a_i\}\colon \bigcup_{i=1}^m (a_i,b_i)\supset (a,b)\}\\
		 			&\geq b-a.
	\end{align*}
	On the other hand, since $[a,b]\subset(a-1/n,b+1/n)$  by all $n$, we have $\mu^*([a,b])\leq b-a$, which is to say
	\[\mu^*[a,b]=b-a.\]
\end{proof}

\begin{pro}%4
	The outer Lebesgue measure of each of the intervals $(a,b),[a,b),(a,b]$ is equal to $b-a$.
\end{pro}
\begin{proof}
	Notice that one-point set is Lebesgue set with measure zero, we can get the conclusion immediately.
\end{proof}

\begin{pro}%5
	Consider the transformation $Tx=ax+b$ from the real line onto itself, where $a,b$ are real numbers and $a\neq 0$. It maps sets $E$ onto sets $T(E)$. Denote by $\mu(\mu^*)$ the Lebesgue measure (outer measure) on the real line. Prove:
	\begin{description}
		\item[(a)] For any set $E$, $\mu^*(T(E))=|a|\mu^*(E)$.
		\item[(b)] $E$ is Lebesgue-measurable if and only if $T(E)$ is Lebesgue-measurable.
		\item[(c)] If $E$ is Lebesgue-measurable, then $\mu(T(E))=|a|\mu(E)$.
	\end{description}
\end{pro}
\begin{proof}
	\begin{description}
		\item[(a)] Given any $E$ and $a\neq 0,b\in\m{R}$, we define
			\[aE=\{ax\colon x\in E\},\,E+b=\{x+b\colon x\in E\},\] then when $a>0$,
			\begin{align*}
				\mu^*(aE)&=\inf\{\sum_n (b_n-a_n)\colon \bigcup_n (a_n,b_n)\supset aE\}\\
				&=a\inf\{\sum_n(\frac{b_n}{a}-\frac{a_n}{a})\colon \bigcup_n \frac{1}{a}(a_n,b_n)\supset E\}\\
				&\geq a\mu^*(E).
			\end{align*}
			The inverse also holds since we can write
			\begin{align*}
				\mu^*(E)&=\mu^*(\frac{1}{a}(aE))\\
				&\geq \frac{1}{a}\mu^*(aE).
			\end{align*}
			And the similar conclusion holds when $a<0$, thus we have $\mu^*(aE)=|a|\mu^*(E)$. Now notice that
			\begin{align*}
				\mu^*(E+b)&=\inf\{\sum_{n}(b_n-a_n)\colon \bigcup_n (a_n,b_n)\supset E+b\}\\
				&=\inf\{\sum_n b_n-a_n\colon \bigcup_n(a_n-b,b_n-b)\supset E\}\\
				&\geq\mu^*(E).
			\end{align*}
			The inverse also holds since we can also write
			\begin{align*}
				\mu^*(E)&=\mu^*(E+b-b)\\
						&\geq\mu^*(E+b)
			\end{align*}
			Thus we have $\mu^*(E)=\mu^*(E+b)$ and finally
			\[\mu^*(aE+b)=|a|\mu^*(E).\]
	\item[(b)] 	Given any $A\subset\m{R}$, there is a $B$, such that $aB+b=A$, thus
	\begin{align*}
		&\phantom{=}\mu^*\big(A\cap(aE+b)\big)\\
		&=\mu^*\big((aB+b)\cap (aE+b)\big)\\
		&=\mu^*\big(a(B\cap E)+b\big)\\
		&=|a|\mu^*(B\cap E).
	\end{align*}
	Similarly we have $\mu^*\big(A\ba(aE+b)\big)=|a|\mu^*(B\ba E)$, thus 
	\begin{align*}
		&\phantom{=}\mu^*\big(A\cap(aE+b)\big)+\mu^*\big(A\ba(aE+b)\big)\\
		&=|a|\mu^*(B\cap E)+|a|\mu^*(B\ba E)\\
		&=|a|\mu^*(B)\\
		&=\mu^*(A).
	\end{align*}
	by supposing that $E$ is a Lebesgue set. Also the inverse deduction is true by writing \[E=\frac{1}{a}(aE+b-b)\].
	\item[(c)] Easy to get from (a) and (b).
	\end{description}
\end{proof}

\section{Metric Spaces}
\begin{pro}%1
	A sequence $\{x_n\}$ is convergent to $y$ if and only if every subsequence $\{x_{n_k}\}$ is convergent to $y$.
\end{pro}
\begin{proof}
	Since $n_{k}\to\infty$ when $k\to\infty$.
\end{proof}

\begin{pro}%2
	A set $D$ is closed if and only if $\overline{D}=D$.
\end{pro}
\begin{proof}
	Suppose $D$ is closed while $D\neq\overline{D}$, then there is some $x\in D'\cap D^c$, then there is some $\delta>0$, such that
	$B(x,\delta)\subset D^c$. Hence $\rho(D,x)\geq \delta>0$, which is to say $x\not\in D'$, so $D=\overline{D}$. Vise versa, suppose $D=\overline{D}$ while $D$ is not closed, i.e. $D^c$ is not open(thus cannot be empty set), so there is some $x\in D^c$(Hence $x\not\in D')$, for all $n\geq 1$,
	$B(x,1/n)\not\subset D^c$. Hence there is a sequence $\{y_n\}$, such that $y_n\in D$ and $0<\rho (y_n,x)<1/n$, that is to say
	$y_n\to x(n\to\infty)$. So we know that $x\in D'$, which is a  contradiction.
\end{proof}

\begin{pro}%3
	$\overline{A\cup B}=\overline{A}\cup\overline{B}$ and $\overline{\overline{A}}=\overline{A}$.
\end{pro}
\begin{proof}
	Notice that for any $x\in(A\cup B)'$, there is some sequence in $A\cup B$ convergent to $x$. So, there is at least one subsequence in at least one set of $A$ and $B$ convergent to $x$, which is to say $x\in A'\cup B'$. By this we can easily know $(A\cup B)'=A'\cup B'$. As for the second one, we notice that
	\[\overline{\overline{A}}=\overline{A\cup A'}=\overline{A}\cup \overline{A'},\]
	so the conclusion is true if and only if 
	\[A\cup A'=\overline{A}\supset \overline{A'}=A'\cup A''.\]
	For any point $x\in A''$ and for any $n\geq 1$, there is some $y_n\in A'$, such that $\rho(x,y_n)<1/(2n)$. Since $y_n$ is in $A'$, there is some $z_n\in A$, such that $\rho(y_n,z_n)<1/(2n)$. Hence $\{z_n\}$ is a sequence in $A$ such that
	\[z_n\to x,\quad (n\to\infty),\]
	which is to say $A''\subset A'$ and finishes our proof.
\end{proof}

\begin{pro}%4
	A set $E$ is open if and only if $E=\mathop{int} E$.
\end{pro}
\begin{proof}
	Trivial.
\end{proof}

\begin{pro}%5
	Let $A$ be any set and $x,y$ be any two points. If $\rho(x,A)>a>b>\rho(y,A)$, then $\rho(x,y)>a-b$.
\end{pro}
\begin{proof}
	We just need to prove $\rho(x,y)+\rho(y,A)\geq \rho(x,A)$. For any $z\in A$, since
	\[\rho(x,y)+\rho(y,z)\geq \rho(x,z),\]
	we have
	\[\rho(x,y)+\rho(y,z)\geq \inf\{\rho(x,t)\colon t\in A\}=\rho(x,A).\]
	Thus
	\[\rho(x,y)+\inf\{\rho(y,t)\colon t\in A\}=\rho(x,y)+\rho(y,A)\geq \rho(x,A).\]
\end{proof}

\begin{pro}%6
	If $A$ is a closed set and $x\not\in A$, then $\rho(x,A)>0$.
\end{pro}
\begin{proof}
	Suppose $\rho(x,A)=0$, then there is a sequence $\{y_n\}$ in $A$, such that $y_n\to x(n\to\infty)$, that is to say  $x\in A'\subset\overline{A}=A$, which is a contradiction.
\end{proof}

\begin{pro}%7
	Denote by $\ma{D}$ the $\sigma$-ring generated by the class of all the closed sets of $X$. Show $\ma{D}=\ma{B}$, where $\ma{B}$ consists of all Borel sets of $X$.
\end{pro}
\begin{proof}
	For any closed set $A\subset X$, since $X$ and $X\ba A$ are both open and $\ma{R}$ is a $\sigma$-ring, we know
	\[A=X\ba(X\ba A)\in \ma{B}.\]
	 Then we know $\ma{D}\subset\ma{B}$ by Problem $1.1.6$. Completely similarly, we have $\ma{B}\subset\ma{D}$, which finished our proof.
\end{proof}

\begin{pro}%8
	Denote by $\ma{P}$ the $\sigma$-ring generated by the class of all half-open intervals $[a,b)$ on the real line. Show that $\ma{P}$ coincides with the class $\ma{B}$ of the Borel sets on the real line.
\end{pro}
\begin{proof}
	Since 
	\[[a,b)=\bigcap_{n} (a-\frac{1}{n},b)\in\ma{B}\]
	by Problem $1.1.3$, we know $\ma{P}\subset \ma{B}$. Vise versa, since
	\[(a,b)=\bigcup_{n} [a+\frac{1}{n},b)\in\ma{P},\]
	we know $\ma{B}\subset\ma{P}$. 
\end{proof}

\section{Metric Outer Measure}
\begin{pro}%1
	Prove the converse of Corollary $1.8.3$-that is, if $\mu^*$ is an outer measure and if every open set is measurable, then $\mu^*$ is a metric outer measure.
\end{pro}
\begin{proof}
	Give any $A,B\subset X$ and $\rho(A,B)=\delta>0$, denote by $U\supset A$ that 
	\[U=\bigcup_{x\in A} B(x,\frac{\delta}{2}).\]
	Easy to know that $B\cap U=\emptyset$ and $U$ is an open set, thus
	\begin{align*}
		\mu^*(A\cup B)&=\mu^*\big((A\cup B)\cap U\big)+\mu^*\big((A\cup B)\ba U\big)\\
		&=\mu^*(A)+\mu^*(B).
	\end{align*}
\end{proof}

\begin{pro}%2
	Let $\mu$ be a measure with domain $\ma{K}$. For any two sets $E$ and $F$ in $\ma{K}$, let $\rho(E,F)=\mu[(E-F)\cup(F-E)].$ Prove that $\rho(E,F)=\rho(F,E)$ and $\rho(E,G)\leq\rho(E,F)+\rho(F,G)$.
\end{pro}
\begin{proof}
	For any $x\in E\ba G$, if $x\in F$, then $x\in E-F$, else $x\in F-G$, thus $E-G\subset(E-F)\cup (F-G)$, which leads to our conclusion.
\end{proof}

\begin{pro}%3
	Let $(X,\rho)$ be a metric space and let $\{x_n\}$ be a sequence of points in $X$. Define $\mu^*(E)$ to be the number of points $x_n$ that belong to $E\subset X$. Prove that $\mu^*$ is a metric outer measure.
\end{pro}
\begin{proof}
	Easy to know $\mu^*$ is an outer measure and $E\cup F=\emptyset$ when $\rho(E,F)>0$, thus $\mu^*$ is a metric outer measure.
\end{proof}

\begin{pro}%4
	Let $(X,\rho)$ be a metric space. Define $\mu^*(E)=1$ if $E\neq \emptyset$ and $\mu^*(\emptyset)=0$. Is $\mu^*$ a metric outer measure?
\end{pro}
\begin{proof}
	From Problem $1.3.2$ we have know $\mu^*$ is an outer measure. Easy to know it is not a metric outer measure.
\end{proof}

\section{Construction of Metric Outer Measures}
\begin{pro}%1
	Let $a_i<a_i+\delta_i<a_i+2\delta_i<\cdots<a_i+r_i\delta_i=b_i$ for $1\leq i\leq n$. Denote by $I_{a,b,t}$ the open interval
	\[a_i+(t_i-1)\delta_i<x_i<a_i+t_i\delta_i,\quad (1\leq i\leq n)\]
	when the $t_i$ vary from $1$ to $r_i$. Let $\lambda$ be defined by 
	\[\lambda(I_{a,b})=\prod_{i=1}^n (b_i-a_i).\]
	Prove that 
	\[\lambda(I_{a,b})=\sum_{t}\lambda(I_{a,b,t}).\]
\end{pro}
\begin{proof}
	Easy to show $\lambda(I_{a,b,t})=\prod_{i=1}^n \delta_i$ and the number of the elements in the sum is $\prod_{i=1}^n r_i$, thus
	\begin{align*}
		\sum_{t}\lambda(I_{a,b,t})
		&=\prod_{i=1}^n (r_i\delta_i)\\
		&=\prod_{i=1}^n (b_i-a_i)\\
		&=\lambda(I_{a,b}).
	\end{align*}
\end{proof}

\begin{pro}%2
	Prove that there is some $\ep_0>0$ and $c$ (both are constant), such that for any $0<\ep<\ep_0$ we have
	\[\la(a-\ep,b+\ep)-\la(a,b)<c\ep.\]
\end{pro}
\begin{proof}
	Notice that
	\begin{align*}
		&\phantom{=}\prod_{i} (b_i-a_i+2\ep)-\prod_{i}(b_i-a_i)\\
		&=M[\prod_{i}(1+\frac{2\ep}{b_i-a_i})-1]\\
		&\leq M[(1+\frac{2\ep}{m})^n-1],
	\end{align*}
	where $M=\prod_i (b_i-a_i)>0$ and $m=\min(b_i-a_i)>0$. From the fact that
	\begin{align*}
		\lim_{x\to 0^+}\frac{(1+x)^n-1}{x}&=\frac{d}{dx}(1+x)^n\bigg|_{x=0}\\
		&=n
	\end{align*}
	we know there is some $\ep_0>0$ such that for all $0<\ep<\ep_0$, \[0<(1+\frac{2\ep}{m})^n-1<(n+1)\frac{2\ep}m.\]
	Thus the conclusion holds.
\end{proof}

\begin{pro}%3
	Prove that every Borel set in $\m{R}^n$ is a Lebesgue set.
\end{pro}
\begin{proof}
	Let $\ma{K}$ be the class of all open intervals in $(a,b)$ form in $\m{R}^n$, and let
	\[\ma{K}_m=\{(a,b)\in \ma{K}\colon d(a,b)\leq 1/m\},\]
	by Theorem $1.9.1,\,1.9.2$ and Corollary $1.8.3$, it suffices to show that $\ma{K_m}$ is good enough according to the condition of Theorem $1.9.2$.
	Given any $(a,b)=\{(\seq{x}{1}{n})\colon a_i<x_i<b_i\}$, we can divide it to many small parts according to Problem $1.9.1$. More specificity, for $m$ which is large enough, we need to let
	\[r_i\delta_i=b_i-a_i,\, \sum_{i=1}^n \delta_i<\frac{1}{m},\]
	then for every $I_{a,b,t}$, its diam is smaller than $1/m$, and the sum of  $\la(I_{a,b,t})$ is exactly $\la(a,b)$ by Problem $1.9.1$. 
	Now it remains to cover those single partition points, which consist the set
	\[J=\{x\colon x_i=a_i+t_i\delta_i,\,1\leq t_i\leq r_i\}.\]
	Easy to know $|J|=\prod_{i} r_i$, and for every point of $J$(i.e. for every $t=(\seq{t}{1}{n})$), we can cover it by
	\[E_j=\{x\colon a_i+t_i\delta_i-\frac{1}{2nm}<x_i<a_i+t_i\delta_i+\frac{1}{2nm}\},\quad(1\leq j\leq |J|).\]
	Then $d(E_j)<1/m$ and
	\[\sum_j\la(E_j)=\big(\frac{1}{nm}\big)^n|J|\leq \frac{|J|}{m}.\]
	For any $\ep>0$, let $m>|J|/\ep$, then the difference between the sum of $\la$ of those covering sets in $\ma{K}_m$ and $\la(a,b)$ is 
	\[\sum_{j}\la(E_j)<\ep,\]
	which is to say the Lebesgue outer measure $\mu^*$ is a metric outer measure, then every Borel set is $\mu^*$-measurable, i.e. is a Lebesgue set.
\end{proof}

\begin{pro}%4
	Let $I_{c_k,e_k}\;(k=1,\ldots,m)$ form a covering of a closed interval $\overline{I}_{a,b}$. Let $r$ be any positive integer and let $\delta_i=(b_i-a_i)/r$. Show that there is a covering of $\overline{I}_{a,b}$ by intervals $I_{\alpha_k,\beta_k}\;(k=1,\ldots,m)$, where $\alpha_k=(\alpha_{k1},\ldots,\alpha_{kn}$ and $\beta_k=(\beta_{k1},\ldots,\beta_{kn})$ have the form
	\begin{align*}
		\alpha_{ki}&=a_i+t_i'\delta_i\quad(t_i'\,\mbox{integer}),\\
		\beta_{ki}&=a_i+t_i''\delta_i\quad(t_i''\,\mbox{integer}),
	\end{align*}
	and $I_{c_k,e_k}\subset I_{\alpha_k,\beta_k}$,
	\[\la(I_{\alpha_k,\beta_k})\leq\la(I_{c_k,e_k})+\frac{C}{r},\]
	where $C$ is independent of $r$.
\end{pro}
	
\begin{pro}%5
	Let $\overline{I}_{a,b}$ be the closure of a bounded interval $I_{a,b}$ in $\m{R}^n$. Show that 
	\[\mu^*(\overline{I}_{a,b})=\mu(\overline{I}_{a,b})=\la(I_{a,b})=\prod{i=1}^n (b_i-a_i).\]
\end{pro}

\begin{pro}%6
	Let $I_{a,b}$ be an open bounded interval in $\m{R}^n$. Show that
	\[\mu(I_{a,b})=\mu^*(I_{a,b})=\mu(\overline{I}_{a,b})=\la(I_{a,b})=\prod_{i=1}^n (b_i-a_i).\]
\end{pro}	

\begin{pro}%7
	If a set $F$ in $\m{R}^n$ is Lebesgue-measurable and if $\mu(F)<\infty$, then for any $\ep>0$ there existsan open set $E$ such that $E\supset F$ and $\mu(E)<\mu(F)+\ep$.
\end{pro}
\begin{proof}
	Recall the definition of Lebesgue outer measure, we know that for any $\ep>0$, there is some $\{E_n\}$ in $\ma{K}$, such that $E_n$ is a bounded open interval and $E=\cup_n E_n\supset F$ and
	\[\sum_{n} \la(E_n)<\mu(F)+\ep.\]
	Since $E$ can be covered by itself and $E$ is also an open set, we know
	\[\mu(E)\leq \sum_{n}\la(E_n),\]
	thus this $E$ is what we need.
\end{proof}

\begin{pro}%8
	If a set $F$ in $\m{R}^n$ is a Lebesgue set, then there exists a Borel set $E$, such that $E\supset F$ and $\mu(E-F)=0$.
\end{pro}
\begin{proof}
	First assume $\mu(F)<\infty$, then according to Problem $1.9.7$, for any $m\geq 1$, there is an open set $E_m$, such that
	$E_m\supset F$ and 
	\[\mu(F)\leq \mu(E_m)<\mu(F)+\frac{1}{m}.\]
	 Now Consider $F_m=\cap_{i=1}^m E_i$, then $F_m\supset F$ and $F_{m+1}\subset F_{m}$, thus we just assume $\{E_m\}$ has the same properties like $\{F_m\}$. Now let $E=\cap_m E_m\supset F$, then $E$ is a Borel set (more specificly, a $G_{\delta}$ set. Cf. Problem $2.15.1$.) and
	\[\mu(E)=\lim_m \mu(E_m)=\mu(F),\]
	since $\mu(E_m)<\infty$, thus $\mu(E-F)=0$. As for $\mu(F)=\infty$, we first divide $\m{R}^n$ to countable many small intervals(every of them is a Borel set with finite measure), and these intervals consist a partition of $\m{R}^n$, call these intervals $\{G_m\}$, then for any $F\cap G_m$(which has a finite measure), there is some Borel set $E_m$(which is not what we create when $\mu(F)<\infty$) such that $E_m\supset F\cap G_m$ and
	\[\mu(E_m-(F\cap G_m))=0.\]
	Now let $E=\cup_m E_m$, then $E\supset \cup_m(F\cap G_m)=F$ is a Borel set and 
	\[\mu(E-F)=\mu\bigg(\bigcup_m \big(E_m-(F\cap G_m)\big)\bigg)=0.\]
	Thus we finish our proof.
\end{proof}

\begin{pro}%9
	A straight line in $\m{R}^n\;(n\geq 2)$ has Lebesgue measure zero. More generally, a $k$-dimensional plane in $\m{R}^n$ ($n\geq k+1$) has LEbesgue measure zero.
\end{pro}
\begin{proof}
	From Problem $1.9.11$ we only need to check
	\[P=\{x\in\m{R}^n\colon x_n=0\}.\]
	First we give a countable partition of $P$, $P=\cup_j G_j$, where $G_j$ is a $n-1$-dimensional rectangle with finite size, then for every $G_j$, there is some open bounded interval $\{E_m\}$ in $\m{R}^n$, such that
	\[E_m\downarrow G_j,\;\mu(E_m)\to 0.\]
	Then $G_j$ is a Borel set according to $\m{R}^n$ and $\mu(G_j)=0$, thus $P$ is also a Borel set and $\mu(P)=0$.
	For any plane whose dimension $k<n-1$, we notice that it can be the intersection of two $k+1$-dimensional plane, thus we know for every $k<n-1$, the $k$-dimensional plane is a Borel set with Lebesgue measure zero.
\end{proof}

\begin{pro}%10
	The boundary of a ball in $\m{R}^n$ has Lebesgue measure zero.
\end{pro}
\begin{proof}
	Easy to know that any bounded open or closed ball is a Borel set with finite Lebesgue measure. For any $0<r,a<\infty$, we can write
	$B_0=B(0,ar)=AB(0,r)$, where $A=\diag\{a,\ldots,a\}$. By Problem $1.9.11$, we know
	\[\mu\big(B(0,ar)\big)=a^n \mu(B_0).\]
	Thus we give a sequence of open ball $B_m=B(0,(1+1/m)r)$ with measure 
	\[\mu(B_m)=(1+1/m)^n \mu(B_0)\to\mu(B_0),\quad (m\to\infty).\]
	On the other hand, $B_m\downarrow \overline{B_0}$, then $\mu(B_m)\to \mu(\overline{B_0})$ and we finally have $\mu(\partial B_0)=0$.
\end{proof}

\begin{pro}%11
	Consider the transformation $Tx=Ax+k$ in $\m{R}^n$, where $A$ is a nonsingular $n\times n$ matrix and $x,k$ are column $n$-vectors.
	$T$ maps sets $E$ onto sets $T(E)$. Assume that $\la(T(a,b))=|\det A|\la(a,b)$.(This will be proved in Problem $2.16.7$ and $2.16.8$) Prove that $T$ satisfies the properties (a)-(c) of Problem $1.6.5$, with $|a|$ replaced by $|\det A|$.
\end{pro}
\begin{proof}
	The similar way by noticing that $A$ is nonsingular.
\end{proof}

\begin{pro}%12
	Extract from the interval $(0,1)$ the middle third-that is, the interval $I_1=(1/3,2/3)$. Next extract from the two remaining intervals, $(0,1/3)$ and $(2/3,1)$, the middle thirds $I_2=(1/9,2/9)$ and $I_3=(7/9,8/9)$, and so on. Let $I=\cup_n I_n$. The set $C=[0,1]\ba I$ is called Cantor's set. Prove that the Lebesgue measure of $C$ is zero.
\end{pro}
\begin{proof}
	For any $n$, let $E_n=[0,1]\ba \cup_{i=1}^n I_i$, then $E_{n+1}\subset E_n$ and $E_n\downarrow C$. Notice that $E_n$ is a set consisting of $2^{n-1}$ disjoint small closed intervals, each with measure $3^{-(n-1)}$. Thus
	 \[\mu(E_n)=(2/3)^{n-1}\to 0=\mu(C),\quad (n\to\infty).\]
\end{proof}

\begin{pro}%13
	The Lebesgue-Stieltjes outer measure $\mu_f^*$, induced by a monotone increasing function $f$(continuous on the right), satisfies
	\[\mu_f^*(a,b]=f(b)-f(a).\]
\end{pro}
\begin{proof}
	First we consider any closed interval $[a,b]$. Given any covering of it by $\cup_n (a_n,b_n)\supset [a,b]$, by Henie-Borel Theorem, there exists a finite sub-covering
	\[\bigcup_{i=1}^m (a_i,b_i)=(\min\{a_i\},\max\{b_i\}).\]
	Thus
	\begin{align*}
		\mu_f^*([a,b])&=\inf\{\sum_n \big(f(b_n)-f(a_n)\big)\colon \bigcup_n (a_n,b_n)\supset [a,b]\}\\
		&\geq\inf\{f(\max\{b_i\})-f(\min\{a_i\})\colon \bigcup_{i=1}^m (a_i,b_i)\supset [a,b]\}\\
		&\geq f(b)-f(a^-),
	\end{align*}
	where $f(a^-)=\lim_{x\to a^-} f(x)$.
	The first $\geq$ is true because $f$ is monotone increasing and the last $\geq$ is true by $f$ is right continuous and a conclusion in Analysis that any monotone function in $\m{R}$ has both left and right limitation at every point. On the other hand, since $[a,b]\subset (a-1/n,b+1/n)$ by all $n$, $\mu_f^*([a,b])\leq \la(a-1/n,b+1/n)=f(b+1/n)-f(a-1/n)$. By taking limitation both sides, we have $\mu_f^*([a,b])\leq f(b)-f(a^-)$, thus
	\[\mu_f^*([a,b])=f(b)-f(a^-).\]
	Now consider any single point set $\{a\}$, it can be covered by $(a-1/n,a+1/n)$ by all $n$, thus $\mu_f^*(\{a\})\leq f(a)-f(a^-)$ and then
	\[f(b)-f(a^-)=\mu_f^*[a,b]\leq \mu_f^*(a,b]+\mu_f^*(\{a\})\leq \mu_f^*(a,b]+f(a)-f(a^-).\]
	Since $f$ is real-valued, we know $\mu_f^*(a,b]\geq f(b)-f(a)$. On the other hand, $(a,b]$ can by covered by $(a,b+1/n)$ by all $n$, thus $\mu_f^*(a,b]\leq f(b)-f(a)$ since $f$ is right continuous. Finally we have
	\begin{align*}
		\mu_f^*(a,b]&=f(b)-f(a),\\
		\mu_f^*(\{a\})&=f(a)-f(a^-).
	\end{align*}
\end{proof}

\begin{pro}%14
	The Lebesgue-Stieltjes outer measure is a metric outer measure.
\end{pro}
\begin{proof}
	 First we prove that for any real-valued monotone increasing right continuous function and for any bounded open interval $(a,b)$, there are at most countable many uncontinuous points in $(a,b)$. Let $\Delta(x)=f(x)-f(x^-)$ and
	 \[E=\{x\in(a,b)\colon \Delta(x)>0\},\;E_n=\{x\in(a,b)\colon \Delta(x)\geq \frac{1}{n}\},\]
	 then $E=\cup_{n=1}^\infty E_n$ is the set of all uncontinuous points. Suppose $E$ is an uncountable set, then there is at least one $n_0$, such that $E_{n_0}$ is an infinite set.(Otherwise, if for all $n$, $E_n$ is finite, then $\cup_n E_n$ is a countable set.) Thus
	 \[f(b)-f(a)\geq \sum_{x\in E_{n_0}}\Delta(x)\geq \sum_{x\in E_{n_0}}\frac{1}{n_0}=\infty,\]
	 which is a contradiction by $f$ is real-valued.
	 Now Assume $E=\{x_n\}$ consist of all uncontinuous points, where $x_n$ are mutually different. Since $\sum_{n} \Delta(x_n)\leq f(b)-f(a)$, so it converges to a finite positive number, then for any $\ep>0$, there is some $N$ large enough, such that
	 \[\sum_{n=N+1}^{\infty} \Delta(x_n)<\frac{\ep}{2}.\]
	 If we denote $F$ by $F=\{x_n\colon 1\leq n\leq N\}$, then for any integer $m$ large enough, we can find
	 \[a=y_0<y_1<\cdots<y_k<y_{k+1}=b\]
	 where $y_i\not\in F$, $y_{i+1}-y_{i}<1/m\;(0\leq i\leq k)$ and here $k$ is depended on $m$ but a finite number. Thus $\{(y_i,y_{i+1})\}\;(0\leq i\leq k)$ can cover $(a,b)$ except for finite single points $y_j\;(1\leq j\leq k)$. Notice that 
	 \[\sum_{i=0}^k \la(y_i,y_{i+1})=f(b)-f(a)=\la(a,b),\] then it just remains to cover those $y_j$s with open intervals whose measures are small enough.
	 For every such $y_j$, since $f$ is right continuous and has a left limit on every point, there is some $\delta_j>0$, such that $2\delta_j<1/m$ and
	 \[f(y_j+\delta_j)<f(y_j)+\frac{\ep}{4k}\;\mbox{and}\;f(y_j-\delta_j)>f(y_j^-)-\frac{\ep}{4k},\]
	 thus
	 \begin{align*}
	 	\la(y_j-\delta_j,y_j+\delta_j)&=f(y_j+\delta_j)-f(y_j-\delta_j)\\
	 	&< f(y_j)-f(y_j^-)+\frac{\ep}{2k}.
	 \end{align*}
	 Then 
	 \begin{align*}
	 \sum_{j=1}^k \la(y_j-\delta_j,y_j+\delta_j)&<\frac{\ep}{2}+\sum_j \Delta(y_j)\\
	 &\leq \frac{\ep}{2}+\sum_{n=N+1}^{\infty} \Delta(x_n)\\
	 &<\ep,
	 \end{align*}
	 and thus L-S outer measure satisfies the condition of Theorem $1.9.2$, then is a metric outer measure according to Theorem $1.9.1$ and $1.9.2$.
\end{proof}

\begin{pro}%15
	Let $f(x)=0$ if $x<0$, $f(x)=1$ if $x\geq 0$. Prove that $\mu_f(-1,0)<1$.
\end{pro}
\begin{proof}
	Since L-S outer measure is a metric outer measure, every Borel set is L-S measurable. For any $(a,b)$,
	\begin{align*}
		\mu_f(a,b)&=\mu(a,b]-\mu(\{b\})\\
		&=(f(b)-f(a))-(f(b)-f(b^-))\\
		&=f(b^-)-f(a),
	\end{align*}
	thus $\mu_f(-1,0)=0$.
\end{proof}

\section{Signed Measures}
\begin{pro}%1
	If $X=\cup_{n=1}^\infty A_n$ and $\mu$ is a signed measure with $\mu(A_n)<\infty$ for all $n$, then $\mu^+$ and $\mu^-$ are $\sigma$-finite.
\end{pro}
\begin{proof}
	Easy to know $\mu^{\pm}(A_n)<\infty$.
\end{proof}

\begin{pro}%2
	Let $\mu$ be a signed measure and let $\{E_n\}$ be a monotone sequence of measurable sets with $|\mu(E_1)|<\infty$ if $\{E_n\}$ is decreasing. Then
	\[\mu(\lim_n E_n)=\lim_n \mu(E_n).\]
\end{pro}
\begin{proof}
	Let $E=\lim_n E_n$, then $\mu^{\pm}(E_n)\to\mu^{\pm}(E)$.
\end{proof}

\begin{pro}%3
	Give an example of a signed measure for which the Hahn decomposition is not unique.
\end{pro}
\begin{proof}
	Let $f_1$ and $f_2$ be two continuous increasing function on $\m{R}$ that
	\[f_1(x)=
	\begin{cases}
		0,&\mbox{if}\; x<0,\\
		x,&\mbox{if}\; 0\leq x\leq 1,\\
		1,&\mbox{if}\; x>1.
	\end{cases}\;\mbox{and}\;
	f_2(x)=
	\begin{cases}
		0,&\mbox{if}\; x<1,\\
		x-1,&\mbox{if}\; 1\leq x\leq 2,\\
		1,&\mbox{if}\; x>2.
	\end{cases}
	\]
 Now let $\mu=\mu_{f_1}-\mu_{f_2}$, then easy to know that both \[A_1=(-\infty,1),B_1=[1,\infty)\; \mbox{and} \;A_2=(0,1),B_2=(-\infty,0]\cup[1,\infty)\] are Hahn decompositions of the signed measure $\mu$.
\end{proof}

\begin{pro}%4
	If $X=A_1\cup B_1=A_2\cup B_2$ are two Hahn decompositions of a signed measure $\mu$, then for any measurable set $E$,
	\[\mu(E\cap A_1)=\mu(E\cap A_2)\;\mbox{and}\;\mu(E\cap B_1)=\mu(E\cap B_2).\]
\end{pro}
\begin{proof}
	Notice that $\mu(E\cap A_1)=\mu(E\cap A_1\cap A_2)+\mu(E\cap A_1\cap B_2)$ and $\mu(E\cap A_1\cap B_2)=0$, then easy to know
	\[\mu(E\cap A_1)=\mu(E\cap A_1\cap A_2)=\mu(E\cap A_2).\]
	Similarly, we know
	\[\mu(E\cap B_1)=\mu(E\cap B_1\cap B_2)=\mu(E\cap B_2).\]
\end{proof}

\begin{pro}%5
	A complex measure is, by definition, a finite complex-valued set function satisfying the properties (i),(iii),(iv) of Definition $1.2.1$. Prove that $\mu$ is a complex measure with domain $\ma{K}$ if and only if there exist finite measures $\mu_1,\ldots,\mu_4$ with the same domain $\ma{K}$ such that
	\[\mu=\mu_1-\mu_2+i(\mu_3-\mu_4).\]
\end{pro}
\begin{proof}
	We only prove the existence of $\mu_1,\ldots,\mu_4$, as for the converse one, it's easy. Assume that $\mu=\mu_r+i\mu_g$, where $\mu_r$ and $\mu_g$ are the real part and imaginary part of $\mu$, then easy to know that $\mu_r,\;\mu_g$ are finite signed measure. Then use the Jordan decomposition of them.
\end{proof}
