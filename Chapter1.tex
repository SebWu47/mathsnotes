\chapter{Polynomial}
\section{Integer Ring and Number Field}

\begin{pro}%1
Denote $\m{Q}[\sqrt{2}]$ by $\{a+b\sqrt{2}\colon a,b\in\m{Q}\}$. Show that $\m{Q}[\sqrt{2}]$ is a number field.
\end{pro}
\begin{proof}
	Assume $a+b\sqrt{2}=0$, if $b\neq 0$, then $a=-b\sqrt{2}$ is rational as well as irrational, thus $a=b=0$. Now it's easy to show that $(a-b\sqrt{2})/(a^2-2b^2)$ is the inverse number of $a+b\sqrt{2}\neq 0$. 
\end{proof}

\begin{pro}%2
	Denote $\m{Z}[\sqrt{2}]$ by $\{m+n\sqrt{2}\colon m,n\in\m{Z}\}$. Show that $\m{Z}[\sqrt{2}]$ is a number ring but not a number field.
\end{pro}
\begin{proof}
	Easy to show it is a ring and $\sqrt{2}$ has no inverse element in it, thus it is not a field.
\end{proof}

\begin{pro}%3
	Suppose $\m{F}$ is a number field, where $a,b,c$ are three arbitrary numbers in it, prove that
	\begin{description}
	\item[(a)] If $a+b=a+c$, then $b=c$;
	\item[(b)] Define $a-b=a+(-b)$, then $a+(b-a)=b$;
	\item[(c)] $a0=0a=0$;
	\item[(d)] $(-1)a=-a$;
	\item[(e)] If $ab=0$, then $a=0$ or $b=0$.
	\end{description}
\end{pro}
\begin{proof}
	For the first one, we add $-a$ to both sides, for the second, we use the commutative law. As for the third one, notice that $a0=a(0+0)=a0+a0$, then use $(1)$. The fourth one can be shown by $(-1)a+a=0a=0$. Finally, the last one is given by $b=a^{-1}0=0$.
\end{proof}

\begin{pro}%4
	Let $F$ is the class of all real number pairs $(a,b)$, then 
	\begin{description}
	\item[(a)] If we define 
		\begin{align*}
		(a,b)+(c,d)&=(a+b,c+d)\\
		(a,b)(c,d)&=(ac,bd)
		\end{align*}
		then is $\m{F}$ a field or not?
	\item[(b)] If we define
		\begin{align*}
		(a,b)+(c,d)&=(a+b,c+d)\\
		(a,b)(c,d)&=(ac-bd,ad+bc)
		\end{align*}
		then is $\m{F}$ a field or not?
	\item[(c)] If $\m{F}$ is the class of all complex number pairs, then what happens to (a) and (b).
	\end{description}
\end{pro}
\begin{proof}
	For the first one, $\m{F}$ is indeed a field, where $(a^{-1},b^{-1})$ is the inverse element of $(a,b)$. For the second one, since $(a,-b)/(a^2+b^2)$ is the inverse element of $(a,b)$, $\m{F}$ is still a field. As for the last one, $\m{F}_1$ is but $\m{F}_2$ is not.
\end{proof}

\begin{pro}%5
	Prove that the commutative law of addition in the definition of commutative ring can be canceled without changing the definition itself.
\end{pro}
\begin{proof}
	First we prove that every element has only one additive inverse element. Assume that $b,c$ are both inverse elements of $a$, then $c=c+(a+b)=(c+a)+b=b$, we denote the inverse element of $a$ by $-a$. Thus if $a+b=a+c=0$, we add both $-a$ to both sides and get $b=c$. Now we prove that $(-1)a=-a$ by $(-1)a+a=(-1)a+1a=(1-1)a=0a=0$. Notice that
	\[(-1)(a+b)=(-1)a+(-1)b=(-a)+(-b),\] and 
	\[(-a)+(-b)+b+a=(-a)+a=0.\]
	Thus we know $a+b=b+a$.
\end{proof}

\section{Polynomial Ring in One Variable}
\begin{pro}%1
	Suppose $f,g\in F[x]$. Show that $f=0$ or $g=0$ if and only if $fg=0$.
\end{pro}
\begin{proof}
	Notice if $fg=0$, then $\deg f+\deg g=\deg(fg)=-\infty$, thus $\deg f=-\infty$ or $\deg g=-\infty$.
\end{proof}

\begin{pro}%2
	Suppose $f,g,h\in F[x]$ and $f\neq 0$. Show that if $fg=fh$, then $g=h$.
\end{pro}
\begin{proof}
	Just notice that $f(g-h)=0$ and thus $g-h=0$.
\end{proof}

\begin{pro}%3
	Suppose the nonzero polynomial $f$ with real coefficients has the property $f(f(x))=f^k(x)$, where $k$ is a given positive integer. Try to find $f$.
\end{pro}
\begin{proof}
	Assume that $n=\deg f$, if $n=0$m then easy to know that $f=1$ when $k$ is even and $f=\pm 1$ when $k$ is odd. Now suppose $n\geq 1$, then easy to know $\deg f(f(x))=n^2$ and $\deg f^k=nk$, thus $n=k$. So we write
	\[f(x)=a_kx^k+a_{k-1}x^{k-1}+\cdots+a_1x+a_0\]
	where $a_k\neq 0$. We define $g_m$ by the coefficient of $x^m$ of any polynomial $g$, then $f(f(x))_{k^2}=a_k^{k+1}$ and $f^k(x)_{k^2}=a_k^k$, thus $a_k=1$. Thus we know
	\[f(f)=f^k+a_{k-1}f^{k-1}+\cdots+a_1f+a_0=f^k\]
	and $g(x)=a_{k-1}f^{k-1}+\cdots+a_1f+a_0=0$.
	Assume that $a_{k-1}\neq 0$, then the coefficient of the maximum order item of $g$ is just $a_{k-1}$, which is a contradiction. Thus $a_{k-1}=0$. Completely similarly, we know that
	\[a_{k-1}=a_{k-2}=\cdots=a_1=a_0=0\] and thus $f(x)=x^k$.
	From all above, 
	\[f(x)=\begin{cases}
	1,x^k,&\mbox{if}\;k\;\mbox{is even},\\
	\pm 1,x^k,&\mbox{if}\;k\;\mbox{is odd}.
	\end{cases}\]
\end{proof}

\begin{pro}%4
	Suppose that the nonzero polynomial $f(x)$ with real coefficients has the property $f(x^2)=f^2(x)$. Try to find $f(x)$.
\end{pro}
\begin{proof}
	Let $n=\deg f\geq 0$. If $n=0$, then $f=c\neq 0$ and $c=c^2$, thus $c=1$. Now assume that $n\geq 1$ and $g(x)=f(x^2)$, then
	\[f(x)=a_nx^n+a_{n-1}x^{n-1}+\cdots+a_1x+a_0,\]
	where $a_n\neq 0$.
	Thus $g_{2n}=a_n$, and $f^2_{2n}=a_n^2$, thus $a_n=1$. Now $g_{2n-1}=0$ and $f^2_{2n-1}=2a_{n-1}$, thus $a_{n-1}=0$. Similarly, we have $g_{2n-2}=a_{n-1}=0$ and $f^2_{2n-2}=2a_{n-2}$, thus $a_{n-2}=0$. Totally similarly, we know that
	\[a_{n-1}=a_{n-2}=\cdots=a_1=a_0=0\]
	and finally $f(x)=x^n$ where $n\geq 0$. 
\end{proof}

\begin{pro}%5
	Suppose the real coefficients polynomial $p(x)=a_0+a_1x+\cdots+a_nx^n$ has the property $0\leq a_0=a_n\leq a_1=a_{n-1}\leq\cdots\leq a_{[n/2]}=a_{[(n+1)/2]}$, denote $A(n)$ by all such $p(x)$s. Show that if $p(x)\in A(n)$ and $q(x)\in A(m)$, then $p(x)q(x)\in A(n+m)$.
\end{pro}

\section{Divisibility and the Greatest Common Divisor}
\begin{pro}%1
	Suppose $f(x)=x^4+3x^2+ax+b$ is divided by $g(x)=x^2-2ax+2$ where $a,b\in\m{R}$, try to find $a$ and $b$.
\end{pro}
\begin{proof}
	Let $f(x)=q(x)g(x)+r(x)$, then $q(x)=x^2+2ax+(1+4a^2)$ and
	\[r(x)=a(8a^2-1)x-2(1+4a^2)+b=0.\]
	Thus we know $(a,b)=(0,1)$ or $(\pm \sqrt{2}/4,3)$.
\end{proof}

\begin{pro}%2
	Show that $f(x)=x^{3m}+x^{3n+1}+x^{3p+2}$ is divided by $g(x)=x^2+x+1$ where $m,n$ and $p$ are all positive integers.
\end{pro}
\begin{proof}
	Suppose that $f(x)=q(x)(x^2+x^1)+r(x)$ where $\deg q(x)\leq 1$. Easy to know there are two different complex number $s_1<s_2$ such that $g(s_1)=g(s_2)=0$. If $g(s)=0$, then $s^3-1=(s-1)(s^2+s+1)=0$, thus $f(s)=1+s+s^2=0$ and we have $r(s_1)=r(s_2)=0$. Assume that $\deg r(x)=0$, then there are exactly one complex number $t$ such that $r(t)=0$, which is a contradiction. Thus $\deg r(x)=-\infty$ and $r(x)=0$.
\end{proof}

\begin{pro}%3
	Show that $(x+y)^n-x^n-y^n$ is divided by $x^2+xy+y^2$ when $n=6m+5$. $(x+y)^n-x^n-y^n$ is divided by $(x^2+xy+y^2)^2$ when $n=6m+1$. Here $x,y$ are both real number and $m$ is any integer making $n>0$.
\end{pro}

\begin{pro}%4
	Find the gcd of $f$ and $g$:
	\begin{description}
	\item[(a)] $f(x)=x^4+x^3-3x^2-4x-1$ and $g(x)=x^3+x^2-x-1$;
	\item[(b)] $f(x)=x^6+2x^4-4x^3-3x^2+8x-5$ and $g(x)=x^5+x^2-x+1$;
	\item[(c)] $f(x)=3x^6-x^5-9x^4-14x^3-11x^2-3x-1$ and $g(x)=3x^5+8x^4+9x^3+15x^2+10x+9$.
	\end{description}
\end{pro}
\begin{proof}
	We only show the first one.
	\begin{align*}
	f(x)&=x^4+x^3-3x^2-4x-1\\
	g(x)&=x^3+x^2-x-1\\
	q_1(x)=x\quad r_1(x)&=-2x^2-3x-1\\
	q_2(x)=-\frac{1}{2}x+\frac{1}{4}\quad r_2(x)&=-\frac{3}{4}x-\frac{3}{4}\\
	q_3(x)=\frac{8}{3}x+\frac{4}{3}\quad r_3(x)&=0
	\end{align*}
	Thus $\gcd(f,g)=x+1$.
\end{proof}

\begin{pro}%5
	Find $u(x)$ and $v(x)$, such that $fu+gv=d(x)$ where $d(x)$ is the gcd of $f$ and $g$.
	\begin{description}
	\item[(a)] $f(x)=x^4+2x^3-x^2-4x-2$ and $g(x)=x^4+x^3-x^2-2x-2$;
	\item[(b)] $f(x)=3x^5+5x^4-16x^3-6x^2-5x-6$ and $g(x)=3x^4-4x^3-x^2-x-2$;
	\item[(c)] $f(x)=3x^3-2x^2+x+2$ and $g(x)=x^2-x+1$;
	\item[(d)] $f(x)=x^4-x^3-4x^2+4x+1$ and $g(x)=x^2-x-1$.
	\end{description}
\end{pro}

\begin{pro}%6
	Find $u(x)$ and $v(x)$, such that $fu+gv=1$ where $d(x)$ is the gcd of $f$ and $g$.
\end{pro}

\begin{pro}%7
	Find $u(x)$ and $v(x)$ in the lowest order, such that
	\begin{description}
	\item[(a)] $(x^4-2x^3-4x^2+6x+1)u(x)+(x^3-5x-3)v(x)=x^4$;
	\item[(b)] $(x^4+2x^3+x+1)u(x)+(x^4+x^3-2x^2+2x-1)v(x)=x^3-2x$.
	\end{description}
\end{pro}
\begin{proof}
	First we give two lemmas, to prove the existence and uniqueness of such $u$ and $v$.
	\begin{lem}
		Assume that two polynomials $f(x)$ and $g(x)$ are relatively prime to each other, and if 
		\[n\geq m>\ell\geq 0\]
		where $\deg f=n,\;\deg g=m,\;\deg h=\ell$ and let
		\[H=\{\big(u(x),v(x)\big)\in F^2[x]\colon f(x)u(x)+g(x)v(x)=h(x)\}\]
		then 
		\begin{description}
			\item[(c)] there is some $(u_0,v_0)\in H$ such that for all $(u,v)\in H$
			\begin{align*}
			0&\leq \deg u_0< \deg u,\\
			0&\leq \deg v_0< \deg v.
			\end{align*}
			or 
			\begin{align*}
			0&\leq \deg u_0= \deg u,\\
			0&\leq \deg v_0= \deg v.
			\end{align*}
			We call such pair a lowest-order pair.
			\item[(d)] there is an unique lowest-order pair $(u_0,v_0)$ in $H$ with property
			\[\deg u_0<m,\quad \deg v_0<n.\]
			\item[(e)] any $(u_1,v_1)\in H$ with property
			\[\deg u_1<m,\quad \deg v_1<n\]
			is a lowest-order pair of $H$.
		\end{description}
	\end{lem}
	\begin{proof}
		\begin{description}
			\item[(c)] Easy to know $H$ is not an empty set. Give any $(u,v)\in H$, assume that $u=0$, then $gv=h\neq 0$, thus $v\neq 0$ and $\deg h=\deg g+\deg v\geq \deg g$, which is a contradiction. Similarly we know that $v\neq 0$, i.e.
			\[\deg u,\deg v\geq 0,\quad \forall (u,v)\in H\]
			Give any $(u_1,v_1),(u_2,v_2)\in H$ where $\deg u_1>\deg u_2$, then
			\begin{align*}
			\deg (fu_1)&=\deg f+\deg u_1\\
					   &>\deg f+\deg u_1\\
					   &=\deg (fu_2).
			\end{align*}
			Thus $\deg (gv_1-h)>\deg (gv_2-h)$. Notice that $\deg (gv_i)\geq \deg g>\deg h$, we know that $\deg (gv_i-h)=\deg (gv_i)$ and finally $\deg v_1>\deg v_2$. Similarly we know if $\deg v_1>\deg v_2$, we have $\deg u_1>\deg u_2$. 
			Thus we can find at least one lowest-order pair.
			\item[(d)] Assume that $(u,v)\in H$, then we have  
			\[v=q_1f+v_1,\quad u=q_2g+u_1,\]
			where $\deg v_1<\deg f=n,\deg u_1<\deg g=m$.Easy to know that $fg(q_1+q_2)=0$ thus we write $q_1=-q_2=q$ and have
			\[u_1=u+qg,\quad v_1=v-qf\]
			and $(u_1,v_1)\in H$, thus any lowest-order pair in $H$ has order less than $m$ and $n$ respectively.
			Now assume that $(u_0,v_0)$ and $(u_1,v_1)$ are both lowest-order pair, then 
			\[f(u_0-u_1)=g(v_1-v_0).\]
			Since $\gcd (f,g)=1$ we know $f|(v_1-v_0)$ and $g|(u_0-u_1)$. Notice that $\deg v_1,\deg v_0<\deg f$, we know that $v_1=v_0$ and similarly $u_0=u_1$, which is the uniqueness of lowest-order pair.
			\item[(e)] Similar to (d).
		\end{description}\qedhere
	\end{proof}
	Back to the original problem, with \textsf{Mathematica} we know for (a)
	\[u(x)=9+9x-7x^2\quad v(x)=3+16x-23x^2+7x,\]
	and for (b)
	\[u(x)=2-7x+3x^2+3x^3\quad v(x)=2+x-6x^2-3x^3.\]
\end{proof}

\begin{pro}%8
	Find a polynomial $f(x)$ with the lowest order such that there are $u$ and $v$
	\[f=u(x-1)^2+2x=v(x-2)^3+3x.\]
\end{pro}
\begin{proof}
	We know that $u(x-1)^2-v(x-2)^3=x$ and from Problem $1.3.7$ $\deg v<2$, thus let $v(x)=ax+b$, then
	\[(ax+b)(x-2)^3+x=u(x-1)^2\]
	Let $x=1$, we have $a+b=1$. Give derivative both sides, we get
	\[a(x-2)^3+3(ax+b)(x-2)^2+1=u'(x-1)^2+2u(x-1)\]
	Let $x=1$, we have $2a+3b=-1$ and thus $a=4,b=-3$. Finally $f=24 - 67 x + 66 x^2 - 27 x^3 + 4 x^4$.
\end{proof}

\begin{pro}%9
	Find a polynomial $f(x)$ with the lowest order such that there are $u$ and $v$
	\begin{align*}f&=u(x^4-2x^3-2x^2+10x-7)+x^2+x+1\\
	&=v(x^4-2x^3-3x^2+13x-10)+2x^2-3.\end{align*}
\end{pro}

\begin{pro}%10
	Suppose $\deg f=2n+1$ where $n$ is some positive integer and $(x-1)^n|f(x)+1$, $(x+1)^n|f(x)-1$, try to find $f(x)$.
\end{pro}
\begin{proof}
	Assume $f(x)=u_0(x)(x-1)^n-1=-v_0(x)(x+1)^n+1$, where $0\geq \deg u_0,\deg v_0<n$. Then
	\[u_0(x)(x-1)^n+v_0(x)(x+1)^n=2.\]
	And thus $v_0(x)=(2-u_0(x)(x-1)^n)(x+1)^{-n}$ and for $0\leq i\leq n-1$, we have
	\[v_0^{(i)}(1)=2^{-(n+i)+1}(-1)^i n(n+1)\cdots(n+i-1).\]
	By Taylor series, we know
	\begin{align*}
	v_0(x)&=\sum_{i=0}^{n-1}\frac{v_0^{(i)}(1)}{i!} (x-1)^i\\
		  &=\sum_{i=0}^{n-1}\binom{n+i-1}{i} 2^{1-(n+i)} (1-x)^i.
	\end{align*}
	Similarly we know what $u_0$ is.
	Easy to know if we let $v(x)=v_0(x)+(ax+b)(x-1)^n$ and $u(x)=u_0(x)-(ax+b)(x+1)^n$ where $a\neq 0$, then $u(x-1)^n+v(x+1)^n=2$ and $\deg f=2n+1$ if we let $f=u(x-1)^n-1=v(x+1)^n+1$. Finally we write 
	\[f(x)=1-(x+1)^n\bigg((ax+b)(x-1)^n+\sum_{i=0}^{n-1}\binom{n+i-1}{i} 2^{1-(n+i)} (1-x)^i\bigg),\] where $a,b\in F$ and $a\neq 0$.
\end{proof}
